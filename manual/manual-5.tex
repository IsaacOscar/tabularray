% -*- coding: utf-8 -*-
% !TEX program = lualatex
\documentclass[oneside]{book}

% -*- coding: utf-8 -*-
% !TEX program = lualatex

\newcommand*{\myversion}{2022D}
\newcommand*{\mylpad}[1]{\ifnum#1<10 0\the#1\else\the#1\fi}

\usepackage[a4paper,margin=2.5cm]{geometry}

\setlength{\parindent}{0pt}
\setlength{\parskip}{4pt plus 1pt minus 1pt}

\usepackage{codehigh} % https://ctan.org/pkg/codehigh
\usepackage{tabularray}
\usepackage{array,multirow,amsmath}
\usepackage{chemmacros,environ}
\usepackage{enumitem}

\usepackage[firstpage=true]{background}
\backgroundsetup{contents={}}

\UseTblrLibrary{amsmath,booktabs,counter,diagbox,functional,siunitx,varwidth}

\usepackage{hyperref}
\hypersetup{
  colorlinks=true,
  urlcolor=blue3,
  linkcolor=blue3,
}

\usepackage{tcolorbox}
\tcbset{sharp corners, boxrule=0.5pt, colback=red9}

\usepackage{float}
%\usepackage{enumerate}

\setcounter{tocdepth}{1}

\newcommand*{\K}[1]{\texttt{#1}}
\newcommand*{\V}[1]{\texttt{#1}}
\newcommand*{\None}{$\times$}

\NewTblrEnviron{newtblr}
\SetTblrOuter[newtblr]{long}
\SetTblrInner[newtblr]{
  hlines = {white}, column{1,2} = {co=1}, colsep = 5pt,
  row{odd} = {brown8}, row{even} = {gray8},
  row{1} = {fg=white, bg=purple2, font=\bfseries\sffamily},
}

\NewTblrEnviron{spectblr}
\SetTblrOuter[spectblr]{long}
\SetTblrInner[spectblr]{
  hlines = {white}, column{2} = {co=1}, colsep = 5pt,
  row{odd} = {brown8}, row{even} = {gray8},
  row{1} = {fg=white, bg=purple2, font=\bfseries\sffamily},
  rowhead = 1,
}

\newcommand{\mywarning}[1]{%
  \begin{tcolorbox}
  The interfaces in this #1 should be seen as
  \textcolor{red3}{\bfseries experimental}
  and are likely to change in future releases, if necessary.
  Don’t use them in important documents.
  \end{tcolorbox}
}

%\renewcommand*{\thefootnote}{*}

\colorlet{highback}{azure9}
\CodeHigh{language=latex/table,style/main=highback,style/code=highback}
\NewCodeHighEnv{code}{style/main=gray9,style/code=gray9}
\NewCodeHighEnv{demo}{style/main=gray9,style/code=gray9,demo}

%\CodeHigh{lite}

\CodeHigh{lite}
\setcounter{chapter}{4}

\begin{document}

\chapter{Use Some Libraries}

\mywarning{chapter}

The \verb!tabularray! package emulates or fixes some commands in other packages.
To avoid potential conflict, you need to enable them with \verb!\UseTblrLibrary! command.

\section{Library \V{booktabs}}

When you write \verb!\UseTblrLibrary{booktabs}!,
\verb!tabularray! package will define commands \verb!\toprule!, \verb!\midrule!,
\verb!\bottomrule! and \verb!\cmidrule! inside \verb!tblr! environment.

\begin{demohigh}
\begin{tblr}{llll}
\toprule
 Alpha   & Beta  & Gamma   & Delta \\
\midrule
 Epsilon & Zeta  & Eta     & Theta \\
\cmidrule{1-3}
 Iota    & Kappa & Lambda  & Mu    \\
\cmidrule{2-4}
 Nu      & Xi    & Omicron & Pi    \\
\bottomrule
\end{tblr}
\end{demohigh}

At this moment, \verb!trim! options for \verb!\cmidrule! command are not supported.
But rule colors are possible just like \verb!\hline! and \verb!\cline! commands.

\begin{demohigh}
\begin{tblr}{llll}
\toprule[purple3]
 Alpha   & Beta  & Gamma   & Delta \\
\midrule[blue3]
 Epsilon & Zeta  & Eta     & Theta \\
\cmidrule[azure3]{1-3}
 Iota    & Kappa & Lambda  & Mu    \\
\cmidrule[azure3]{2-4}
 Nu      & Xi    & Omicron & Pi    \\
\bottomrule[purple3]
\end{tblr}
\end{demohigh}

\section{Library \V{diagbox}}

When writing \verb!\UseTblrLibrary{diagbox}! in the preamble of the document,
\verb!tabularray! package loads \verb!diagbox! package,
and you can use \verb!\diagbox! and \verb!\diagboxthree! commands inside \verb!tblr! environment.

\begin{demohigh}
\begin{tblr}{hlines,vlines}
 \diagbox{Aa}{Pp} & Beta & Gamma \\
 Epsilon & Zeta  & Eta \\
 Iota    & Kappa & Lambda \\
\end{tblr}
\end{demohigh}

\begin{demohigh}
\begin{tblr}{hlines,vlines}
 \diagboxthree{Aa}{Pp}{Hh} & Beta & Gamma \\
 Epsilon & Zeta  & Eta \\
 Iota    & Kappa & Lambda \\
\end{tblr}
\end{demohigh}

You can also use \verb!\diagbox! and \verb!\diagboxthree! commands in math mode.
\nopagebreak
\begin{demohigh}
$\begin{tblr}{|c|cc|}
\hline
 \diagbox{X_1}{X_2} & 0 & 1 \\
\hline
  0 & 0.1 & 0.2 \\
  1 & 0.3 & 0.4 \\
\hline
\end{tblr}$
\end{demohigh}

\section{Library \V{siunitx}}

When writing \verb!\UseTblrLibrary{siunitx}! in the preamble of the document,
\verb!tabularray! package loads \verb!siunitx! package,
and defines \verb!S! column for \verb!tblr! environment.

\begin{demohigh}
\begin{tblr}{
  hlines, vlines,
  colspec={
    S[table-format=2.2]
    S[table-format=2.2]
    S[table-format=2.2]
  }
}
 {{{Head}}} & {{{Head}}} & {{{Head}}} \\
    11      &    11      &    11      \\
     2.1    &     2.2    &     2.3    \\
    33.11   &    33.22   &    33.33   \\
\end{tblr}
\end{demohigh}

Note that you need to use \underline{triple} pairs of braces to guard non-numeric cells.
If you need to apply other specifications for the \verb!S! columns,
you need to move \verb!siunitx! options into \verb!si! keys of the \verb!Q! columns.

\begin{demohigh}
\begin{tblr}{
  hlines, vlines,
  colspec={
    Q[si={table-format=2.2},blue7]
    Q[si={table-format=2.2},teal7]
    Q[si={table-format=2.2},purple7]
  }
}
 {{{Head}}} & {{{Head}}} & {{{Head}}} \\
    11      &    11      &    11      \\
     2.1    &     2.2    &     2.3    \\
    33.11   &    33.22   &    33.33   \\
\end{tblr}
\end{demohigh}

\end{document}
