% -*- coding: utf-8 -*-
% !TEX program = lualatex
\documentclass[oneside]{book}

% -*- coding: utf-8 -*-
% !TEX program = lualatex

\newcommand*{\myversion}{2022D}
\newcommand*{\mylpad}[1]{\ifnum#1<10 0\the#1\else\the#1\fi}

\usepackage[a4paper,margin=2.5cm]{geometry}

\setlength{\parindent}{0pt}
\setlength{\parskip}{4pt plus 1pt minus 1pt}

\usepackage{codehigh} % https://ctan.org/pkg/codehigh
\usepackage{tabularray}
\usepackage{array,multirow,amsmath}
\usepackage{chemmacros,environ}
\usepackage{enumitem}

\usepackage[firstpage=true]{background}
\backgroundsetup{contents={}}

\UseTblrLibrary{amsmath,booktabs,counter,diagbox,functional,siunitx,varwidth}

\usepackage{hyperref}
\hypersetup{
  colorlinks=true,
  urlcolor=blue3,
  linkcolor=blue3,
}

\usepackage{tcolorbox}
\tcbset{sharp corners, boxrule=0.5pt, colback=red9}

\usepackage{float}
%\usepackage{enumerate}

\setcounter{tocdepth}{1}

\newcommand*{\K}[1]{\texttt{#1}}
\newcommand*{\V}[1]{\texttt{#1}}
\newcommand*{\None}{$\times$}

\NewTblrEnviron{newtblr}
\SetTblrOuter[newtblr]{long}
\SetTblrInner[newtblr]{
  hlines = {white}, column{1,2} = {co=1}, colsep = 5pt,
  row{odd} = {brown8}, row{even} = {gray8},
  row{1} = {fg=white, bg=purple2, font=\bfseries\sffamily},
}

\NewTblrEnviron{spectblr}
\SetTblrOuter[spectblr]{long}
\SetTblrInner[spectblr]{
  hlines = {white}, column{2} = {co=1}, colsep = 5pt,
  row{odd} = {brown8}, row{even} = {gray8},
  row{1} = {fg=white, bg=purple2, font=\bfseries\sffamily},
  rowhead = 1,
}

\newcommand{\mywarning}[1]{%
  \begin{tcolorbox}
  The interfaces in this #1 should be seen as
  \textcolor{red3}{\bfseries experimental}
  and are likely to change in future releases, if necessary.
  Don’t use them in important documents.
  \end{tcolorbox}
}

%\renewcommand*{\thefootnote}{*}

\colorlet{highback}{azure9}
\CodeHigh{language=latex/table,style/main=highback,style/code=highback}
\NewCodeHighEnv{code}{style/main=gray9,style/code=gray9}
\NewCodeHighEnv{demo}{style/main=gray9,style/code=gray9,demo}

%\CodeHigh{lite}

\CodeHigh{lite}
\setcounter{chapter}{4}

\begin{document}

\chapter{Use Some Libraries}

The \verb!tabularray! package emulates or fixes some commands in other packages.
To avoid potential conflict, you need to enable them with \verb!\UseTblrLibrary! command.

\section{Library \texttt{amsmath}}

With \verb!\UseTblrLibrary{amsmath}! in the preamble of the document,
\verb!tabularray! will load \verb!amsmath! package, and define \verb!+array!, \verb!+matrix!,
\verb!+bmatrix!, \verb!+Bmatrix!, \verb!+pmatrix!, \verb!+vmatrix!, \verb!+Vmatrix! and \verb!+cases!
environments. Each of the environments is similar to the environment without \verb!+! prefix in its name,
but has default \verb!rowsep=2pt! just as \verb!tblr! environment. Every environment
except \verb!+array! accepts an optional argument, where you can write inner specifications.

\begin{demo}
$\begin{pmatrix}
 \dfrac{2}{3} &  \dfrac{2}{3} &  \dfrac{1}{3} \\
 \dfrac{2}{3} & -\dfrac{1}{3} & -\dfrac{2}{3} \\
 \dfrac{1}{3} & -\dfrac{2}{3} &  \dfrac{2}{3} \\
\end{pmatrix}$
\end{demo}

\begin{demohigh}
$\begin{+pmatrix}[cells={r},row{2}={purple8}]
 \dfrac{2}{3} &  \dfrac{2}{3} &  \dfrac{1}{3} \\
 \dfrac{2}{3} & -\dfrac{1}{3} & -\dfrac{2}{3} \\
 \dfrac{1}{3} & -\dfrac{2}{3} &  \dfrac{2}{3} \\
\end{+pmatrix}$
\end{demohigh}

\begin{demo}
$f(x)=\begin{cases}
 0,            & x=1; \\
 \dfrac{1}{3}, & x=2; \\
 \dfrac{2}{3}, & x=3; \\
 1,            & x=4. \\
\end{cases}$
\end{demo}

\begin{demohigh}
$f(x)=\begin{+cases}
 0,            & x=1; \\
 \dfrac{1}{3}, & x=2; \\
 \dfrac{2}{3}, & x=3; \\
 1,            & x=4. \\
\end{+cases}$
\end{demohigh}

\section{Library \texttt{booktabs}}

With \verb!\UseTblrLibrary{booktabs}! in the preamble of the document,
\verb!tabularray! will load \verb!booktabs! package,
and define \verb!\toprule!, \verb!\midrule!,
\verb!\bottomrule! and \verb!\cmidrule! inside \verb!tblr! environment.

\begin{demohigh}
\begin{tblr}{llll}
\toprule
 Alpha   & Beta  & Gamma   & Delta \\
\midrule
 Epsilon & Zeta  & Eta     & Theta \\
\cmidrule{1-3}
 Iota    & Kappa & Lambda  & Mu    \\
\cmidrule{2-4}
 Nu      & Xi    & Omicron & Pi    \\
\bottomrule
\end{tblr}
\end{demohigh}

Just like \verb!\hline! and \verb!\cline! commands,
you can also specify rule width and color in the optional argument of any of these commands.

\begin{demohigh}
\begin{tblr}{llll}
\toprule[2pt,purple3]
 Alpha   & Beta  & Gamma  & Delta \\
\midrule[blue3]
 Epsilon & Zeta  & Eta    & Theta \\
\cmidrule[azure3]{2-3}
 Iota    & Kappa & Lambda & Mu    \\
\bottomrule[2pt,purple3]
\end{tblr}
\end{demohigh}

If you need more than one cmidrules, you can use \verb!\cmidrulemore! command.

\begin{demohigh}
\begin{tblr}{llll}
\toprule
 Alpha   & Beta  & Gamma   & Delta \\
\cmidrule{1-3} \cmidrulemore{2-4}
 Epsilon & Zeta  & Eta     & Theta \\
\cmidrule{1-3} \morecmidrules \cmidrule{2-4}
 Iota    & Kappa & Lambda  & Mu    \\
\bottomrule
\end{tblr}
\end{demohigh}

From version \verb!2021N!, trim options (\verb!l!, \verb!r!, \verb!lr!)
for \verb!\cmidrule! command are also supported.

\begin{demohigh}
\begin{tblr}{llll}
\toprule
 Alpha   & Beta  & Gamma   & Delta \\
\cmidrule[lr]{1-2} \cmidrule[lr=-0.4]{3-4}
 Epsilon & Zeta  & Eta     & Theta \\
\cmidrule[r]{1-2} \cmidrule[l]{3-4}
 Iota    & Kappa & Lambda  & Mu    \\
\bottomrule
\end{tblr}
\end{demohigh}

Note that you need to put \verb!l!, \verb!r! or \verb!lr! option into
the \underline{\color{red3}square brackets}.
and the possible values are decimal numbers between \verb!-1! and \verb!0!,
where \verb!-1! means trimming the whole colsep, and \verb!0! means no trimming.
The default value is \verb!-0.8!, which makes similar result as \verb!booktabs! package does.

There is also a \verb!booktabs! environment for you. With this environment,
the default \verb!rowsep=0pt!, but extra vertical space will be added by
\verb!\toprule!, \verb!\midrule!, \verb!\bottomrule! and \verb!\cmidrule! commands.
The sizes of vertical space are determined by \verb!\aboverulesep! and \verb!\belowrulesep! dimensions.

\begin{demohigh}
\begin{booktabs}{
  colspec = lcccc,
  cell{1}{1} = {r=2}{}, cell{1}{2,4} = {c=2}{},
}
\toprule
  Sample & I &   & II &   \\
\cmidrule[lr]{2-3} \cmidrule[lr]{4-5}
         & A & B & C & D \\
\midrule
  S1     & 5 & 6 & 7 & 8 \\
  S2     & 6 & 7 & 8 & 5 \\
  S3     & 7 & 8 & 5 & 6 \\
\bottomrule
\end{booktabs}
\end{demohigh}
% S4     & 8 & 5 & 6 & 7 \\

You can also use \verb!\specialrule! command.
The second argument sets \verb!belowsep! of previous row,
and the third argument sets \verb!abovesep! of current row,

\begin{demohigh}
\begin{booktabs}{row{2}={olive9}}
\toprule
 Alpha   & Beta  & Gamma   & Delta \\
\specialrule{0.5pt}{4pt}{6pt}
 Epsilon & Zeta  & Eta     & Theta \\
\specialrule{0.8pt,blue3}{3pt}{2pt}
 Iota    & Kappa & Lambda  & Mu    \\
\bottomrule
\end{booktabs}
\end{demohigh}

At last, there is also an \verb!\addlinespace! command.
You can specify the size of vertical space to be added in its optional argument,
and the default size is \verb!0.5em!.
This command adds one half of the space to \verb!belowsep! of previous row,
and the other half to \verb!abovesep! of current row.

\begin{demohigh}
\begin{booktabs}{row{2}={olive9}}
\toprule
 Alpha   & Beta  & Gamma   & Delta \\
\addlinespace
 Epsilon & Zeta  & Eta     & Theta \\
\addlinespace[1em]
 Iota    & Kappa & Lambda  & Mu    \\
\bottomrule
\end{booktabs}
\end{demohigh}

From version 2022A, there is a \verb!longtabs! environment for writing long \verb!booktabs! tables,
and a \verb!talltabs! environment for writing tall \verb!booktabs! tables.

\section{Library \texttt{counter}}

You need to load \verb!counter! library with \verb!\UseTblrLibrary{counter}!,
if you want to modify some LaTeX counters inside \verb!tabularray! tables.

\begin{demohigh}
\newcounter{mycnta}
\newcommand{\mycnta}{\stepcounter{mycnta}\arabic{mycnta}}
\begin{tblr}{hlines}
  \mycnta & \mycnta & \mycnta \\
  \mycnta & \mycnta & \mycnta \\
  \mycnta & \mycnta & \mycnta \\
\end{tblr}
\end{demohigh}

\section{Library \texttt{diagbox}}

When writing \verb!\UseTblrLibrary{diagbox}! in the preamble of the document,
\verb!tabularray! package loads \verb!diagbox! package,
and you can use \verb!\diagbox! and \verb!\diagboxthree! commands inside \verb!tblr! environment.

\begin{demohigh}
\begin{tblr}{hlines,vlines}
 \diagbox{Aa}{Pp} & Beta & Gamma \\
 Epsilon & Zeta  & Eta \\
 Iota    & Kappa & Lambda \\
\end{tblr}
\end{demohigh}

\begin{demohigh}
\begin{tblr}{hlines,vlines}
 \diagboxthree{Aa}{Pp}{Hh} & Beta & Gamma \\
 Epsilon & Zeta  & Eta \\
 Iota    & Kappa & Lambda \\
\end{tblr}
\end{demohigh}

You can also use \verb!\diagbox! and \verb!\diagboxthree! commands in math mode.
\nopagebreak
\begin{demohigh}
$\begin{tblr}{|c|cc|}
\hline
 \diagbox{X_1}{X_2} & 0 & 1 \\
\hline
  0 & 0.1 & 0.2 \\
  1 & 0.3 & 0.4 \\
\hline
\end{tblr}$
\end{demohigh}

\section{Library \texttt{siunitx}}

When writing \verb!\UseTblrLibrary{siunitx}! in the preamble of the document,
\verb!tabularray! package loads \verb!siunitx! package,
and defines \verb!S! column as \verb!Q! column with \verb!si! key.

\begin{demohigh}
\begin{tblr}{
  hlines, vlines,
  colspec={S[table-format=3.2]S[table-format=3.2]},
}
 {{{Head}}} & {{{Head}}} \\
   111      &   111      \\
     2.1    &     2.2    \\
    33.11   &    33.22   \\
\end{tblr}
\end{demohigh}

\begin{demohigh}
\begin{tblr}{
  hlines, vlines,
  colspec={Q[si={table-format=3.2},c]Q[si={table-format=3.2},c]}
}
 {{{Head}}} & {{{Head}}} \\
   111      &   111      \\
     2.1    &     2.2    \\
    33.11   &    33.22   \\
\end{tblr}
\end{demohigh}

Note that you need to use \underline{\color{red3}triple} pairs of braces to guard non-numeric cells.

Also you must use \verb!l!, \verb!c! or \verb!r! to set horizontal alignment for non-numeric cells:
\nopagebreak
\begin{demohigh}
\begin{tblr}{
  hlines, vlines, columns={6em},
  colspec={
    Q[si={table-format=3.2,table-number-alignment=left},l,blue7]
    Q[si={table-format=3.2,table-number-alignment=center},c,teal7]
    Q[si={table-format=3.2,table-number-alignment=right},r,purple7]
  }
}
 {{{Head}}} & {{{Head}}} & {{{Head}}} \\
   111      &   111      &   111      \\
     2.1    &     2.2    &     2.3    \\
    33.11   &    33.22   &    33.33   \\
\end{tblr}
\end{demohigh}

Both \verb!S! and \verb!s! columns are supported. In fact, These two columns have been defined as follows:
\begin{codehigh}
\NewColumnType{S}[1][]{Q[si={#1},c]}
\NewColumnType{s}[1][]{Q[si={#1},c,cmd=\TblrUnit]}
\end{codehigh}
You don't need to and are not allowed to define them again.

\section{Library \texttt{varwidth}}

To build a nice table, \verb!tabularray! need to measure the widths of cells.
By default, it uses \verb!\hbox! to measure the sizes.
This causes an error if a cell contains some vertical material, such as lists or display maths.

With \verb!\UseTblrLibrary{varwidth}! in the preamble of the document,
\verb!tabularray! will load \verb!varwidth! package,
and add a new inner specification \verb!measure! for tables.
After setting \verb!measure=vbox!, it will use \verb!\vbox! to measure cell widths.

\begin{demohigh}
\begin{tblr}{hlines,measure=vbox}
  Text Text Text Text Text Text Text
  \begin{itemize}
    \item List List List List List List
    \item List List List List List List List
  \end{itemize}
  Text Text Text Text Text Text Text \\
\end{tblr}
\end{demohigh}

From version 2022A, you can remove extra space above and below lists,
by adding option \verb!stretch=-1!.
The following example also needs \verb!enumitem! package and its \verb!nosep! option:

{\centering\begin{tblr}{
  hlines,vlines,rowspec={Q[l,t]Q[l,b]},
  measure=vbox,stretch=-1,
}
  \begin{itemize}[nosep]
    \item List List List List List
    \item List List List List List List
  \end{itemize} & oooo \\
  \begin{itemize}[nosep]
    \item List List List List List
    \item List List List List List List
  \end{itemize} & gggg \\
\end{tblr}\par}

%% BUG: there is extra vertical space at the beginning of the first cell if I use demohigh
%\begin{demohigh}
\begin{codehigh}
\begin{tblr}{
  hlines,vlines,rowspec={Q[l,t]Q[l,b]},
  measure=vbox,stretch=-1,
}
  \begin{itemize}[nosep]
    \item List List List List List
    \item List List List List List List
  \end{itemize} & oooo \\
  \begin{itemize}[nosep]
    \item List List List List List
    \item List List List List List List
  \end{itemize} & gggg \\
\end{tblr}
\end{codehigh}
%\end{demohigh}

Note that option \verb!stretch=-1! also removes struts from cells, therefore it may not work well
in \verb!tabularray! environments with \verb!rowsep=0pt!, such as
\verb!booktabs!/\verb!longtabs!/\verb!talltabs! environments from \verb!booktabs! library.

\end{document}
