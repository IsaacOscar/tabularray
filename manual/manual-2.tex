% -*- coding: utf-8 -*-
% !TEX program = lualatex
\documentclass[oneside]{book}

% -*- coding: utf-8 -*-
% !TEX program = lualatex

\newcommand*{\myversion}{2022D}
\newcommand*{\mylpad}[1]{\ifnum#1<10 0\the#1\else\the#1\fi}

\usepackage[a4paper,margin=2.5cm]{geometry}

\setlength{\parindent}{0pt}
\setlength{\parskip}{4pt plus 1pt minus 1pt}

\usepackage{codehigh} % https://ctan.org/pkg/codehigh
\usepackage{tabularray}
\usepackage{array,multirow,amsmath}
\usepackage{chemmacros,environ}
\usepackage{enumitem}

\usepackage[firstpage=true]{background}
\backgroundsetup{contents={}}

\UseTblrLibrary{amsmath,booktabs,counter,diagbox,functional,siunitx,varwidth}

\usepackage{hyperref}
\hypersetup{
  colorlinks=true,
  urlcolor=blue3,
  linkcolor=blue3,
}

\usepackage{tcolorbox}
\tcbset{sharp corners, boxrule=0.5pt, colback=red9}

\usepackage{float}
%\usepackage{enumerate}

\setcounter{tocdepth}{1}

\newcommand*{\K}[1]{\texttt{#1}}
\newcommand*{\V}[1]{\texttt{#1}}
\newcommand*{\None}{$\times$}

\NewTblrEnviron{newtblr}
\SetTblrOuter[newtblr]{long}
\SetTblrInner[newtblr]{
  hlines = {white}, column{1,2} = {co=1}, colsep = 5pt,
  row{odd} = {brown8}, row{even} = {gray8},
  row{1} = {fg=white, bg=purple2, font=\bfseries\sffamily},
}

\NewTblrEnviron{spectblr}
\SetTblrOuter[spectblr]{long}
\SetTblrInner[spectblr]{
  hlines = {white}, column{2} = {co=1}, colsep = 5pt,
  row{odd} = {brown8}, row{even} = {gray8},
  row{1} = {fg=white, bg=purple2, font=\bfseries\sffamily},
  rowhead = 1,
}

\newcommand{\mywarning}[1]{%
  \begin{tcolorbox}
  The interfaces in this #1 should be seen as
  \textcolor{red3}{\bfseries experimental}
  and are likely to change in future releases, if necessary.
  Don’t use them in important documents.
  \end{tcolorbox}
}

%\renewcommand*{\thefootnote}{*}

\colorlet{highback}{azure9}
\CodeHigh{language=latex/table,style/main=highback,style/code=highback}
\NewCodeHighEnv{code}{style/main=gray9,style/code=gray9}
\NewCodeHighEnv{demo}{style/main=gray9,style/code=gray9,demo}

%\CodeHigh{lite}

\CodeHigh{lite}
\setcounter{chapter}{1}

\begin{document}

\chapter{The Old Interfaces}

With tabularray package, you can still use improved table commands to change the styles of tables.
Same as \verb!tabular! and \verb!array! environments,
all table commands \textcolor{red3}{must} be put at the beginning ot the cell text.
Also, new table commands \textcolor{red3}{must} be defined with \verb!\NewTableCommand!.

\section{Hline Commands}

The \verb!\hline! command has an optional argument which accepts key-value options.
The available keys are described in Table \ref{key:hvline}.

\begin{demohigh}
\begin{tblr}{llll}
\hline
 Alpha   & Beta  & Gamma  & Delta \\
\hline[dashed]
 Epsilon & Zeta  & Eta    & Theta \\
\hline[dotted]
 Iota    & Kappa & Lambda & Mu    \\
\hline[2pt,blue5]
\end{tblr}
\end{demohigh}

The \verb!\cline! command also has an optional argument which is the same as \verb!\hline!.

\begin{demohigh}
\begin{tblr}{llll}
\cline{1-4}
 Alpha   & Beta  & Gamma  & Delta \\
\cline[dashed]{1,3}
 Epsilon & Zeta  & Eta    & Theta \\
\cline[dashed]{2,4}
 Iota    & Kappa & Lambda & Mu    \\
\cline[2pt,blue5]{-}
\end{tblr}
\end{demohigh}

You can use child selectors in the mandatory argument of \verb!\cline!.

\begin{demohigh}
\begin{tblr}{llll}
\cline{1-4}
 Alpha   & Beta  & Gamma  & Delta \\
\cline[dashed]{odd}
 Epsilon & Zeta  & Eta    & Theta \\
\cline[dashed]{even}
 Iota    & Kappa & Lambda & Mu    \\
\cline[2pt,blue5]{-}
\end{tblr}
\end{demohigh}

\section{Cell Commands}

The \verb!\SetCell! command has a mandatory argument for setting the styles of current cell.
The available keys are described in Table \ref{key:cell}.

\begin{demohigh}
\begin{tblr}{llll}
\hline[1pt]
 Alpha   & \SetCell{bg=teal2,fg=white} Beta & Gamma & Delta \\
\hline
 Epsilon & Zeta & \SetCell{r,font=\scshape} Eta & Theta \\
\hline
 Iota    & Kappa & Lambda & Mu    \\
\hline[1pt]
\end{tblr}
\end{demohigh}

The \verb!\SetCell! command also has an optional argument for setting the multispan of current cell.
The available keys are described in Table \ref{key:cellspan}.

\begin{demohigh}
\begin{tblr}{|X|X|X|X|X|X|}
\hline
 Alpha & Beta & Gamma & Delta & Epsilon & Zeta \\
\hline
 \SetCell[c=2]{c} Eta & 2-2
              & \SetCell[c=2]{c} Iota & 2-4
                              & \SetCell[c=2]{c} Lambda  & 2-6 \\
\hline
 \SetCell[c=3]{l} Nu & 3-2 & 3-3
                      & \SetCell[c=3]{l} Pi & 3-5 & 3-6   \\
\hline
 \SetCell[c=6]{r} Tau & 4-2 & 4-3 & 4-4 & 4-5 & 4-6 \\
\hline
\end{tblr}
\end{demohigh}

\section{Row Commands}

The \verb!\SetRow! command has a mandatory argument for setting the styles of current row.
The available keys are described in Table \ref{key:row}.

\begin{demohigh}
\begin{tblr}{llll}
\hline[1pt]
 \SetRow{azure8} Alpha & Beta & Gamma & Delta \\
\hline
 \SetRow{blue8,c} Epsilon & Zeta & Eta & Theta \\
\hline
 \SetRow{violet8} Iota & Kappa & Lambda & Mu \\
\hline[1pt]
\end{tblr}
\end{demohigh}

\section{Column Types}

The \verb!tabularray! package has only one type of primitive column: the \verb!Q! column.
Other types of columns are defined as \verb!Q! columns with some keys.

\begin{codehigh}
\NewColumnType{l}{Q[l]}
\NewColumnType{c}{Q[c]}
\NewColumnType{r}{Q[r]}
\NewColumnType{t}[1]{Q[t,wd=#1]}
\NewColumnType{m}[1]{Q[m,wd=#1]}
\NewColumnType{b}[1]{Q[b,wd=#1]}
\NewColumnType{h}[1]{Q[h,wd=#1]}
\NewColumnType{f}[1]{Q[f,wd=#1]}
\NewColumnType{X}[1][]{Q[co=1,#1]}
\end{codehigh}

\begin{demohigh}
\begin{tblr}{|t{15mm}|m{15mm}|b{20mm}|}
 Alpha   & Beta  & {Gamma\\Gamma} \\
 Epsilon & Zeta  & {Eta\\Eta} \\
 Iota    & Kappa & {Lambda\\Lambda} \\
\end{tblr}
\end{demohigh}

Any new column type must be defined with \verb!\NewColumnType! command.
It can have an optional argument when it's defined.

\section{Row Types}

The \verb!tabularray! package has only one type of primitive row: the \verb!Q! row.
Other types of rows are defined as \verb!Q! rows with some keys.

\begin{codehigh}
\NewRowType{l}{Q[l]}
\NewRowType{c}{Q[c]}
\NewRowType{r}{Q[r]}
\NewRowType{t}[1]{Q[t,ht=#1]}
\NewRowType{m}[1]{Q[m,ht=#1]}
\NewRowType{b}[1]{Q[b,ht=#1]}
\NewRowType{h}[1]{Q[h,ht=#1]}
\NewRowType{f}[1]{Q[f,ht=#1]}
\end{codehigh}

\begin{demohigh}
\begin{tblr}{rowspec={|t{12mm}|m{10mm}|b{10mm}|}}
 Alpha   & Beta  & {Gamma\\Gamma} \\
 Epsilon & Zeta  & {Eta\\Eta} \\
 Iota    & Kappa & {Lambda\\Lambda} \\
\end{tblr}
\end{demohigh}

Any new row type must be defined with \verb!\NewRowType! command.
It can have an optional argument when it's defined.

\end{document}
