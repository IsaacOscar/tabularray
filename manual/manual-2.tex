% -*- coding: utf-8 -*-
% !TEX program = lualatex
\documentclass[oneside]{book}

% -*- coding: utf-8 -*-
% !TEX program = lualatex

\newcommand*{\myversion}{2022D}
\newcommand*{\mylpad}[1]{\ifnum#1<10 0\the#1\else\the#1\fi}

\usepackage[a4paper,margin=2.5cm]{geometry}

\setlength{\parindent}{0pt}
\setlength{\parskip}{4pt plus 1pt minus 1pt}

\usepackage{codehigh} % https://ctan.org/pkg/codehigh
\usepackage{tabularray}
\usepackage{array,multirow,amsmath}
\usepackage{chemmacros,environ}
\usepackage{enumitem}

\usepackage[firstpage=true]{background}
\backgroundsetup{contents={}}

\UseTblrLibrary{amsmath,booktabs,counter,diagbox,functional,siunitx,varwidth}

\usepackage{hyperref}
\hypersetup{
  colorlinks=true,
  urlcolor=blue3,
  linkcolor=blue3,
}

\usepackage{tcolorbox}
\tcbset{sharp corners, boxrule=0.5pt, colback=red9}

\usepackage{float}
%\usepackage{enumerate}

\setcounter{tocdepth}{1}

\newcommand*{\K}[1]{\texttt{#1}}
\newcommand*{\V}[1]{\texttt{#1}}
\newcommand*{\None}{$\times$}

\NewTblrEnviron{newtblr}
\SetTblrOuter[newtblr]{long}
\SetTblrInner[newtblr]{
  hlines = {white}, column{1,2} = {co=1}, colsep = 5pt,
  row{odd} = {brown8}, row{even} = {gray8},
  row{1} = {fg=white, bg=purple2, font=\bfseries\sffamily},
}

\NewTblrEnviron{spectblr}
\SetTblrOuter[spectblr]{long}
\SetTblrInner[spectblr]{
  hlines = {white}, column{2} = {co=1}, colsep = 5pt,
  row{odd} = {brown8}, row{even} = {gray8},
  row{1} = {fg=white, bg=purple2, font=\bfseries\sffamily},
  rowhead = 1,
}

\newcommand{\mywarning}[1]{%
  \begin{tcolorbox}
  The interfaces in this #1 should be seen as
  \textcolor{red3}{\bfseries experimental}
  and are likely to change in future releases, if necessary.
  Don’t use them in important documents.
  \end{tcolorbox}
}

%\renewcommand*{\thefootnote}{*}

\colorlet{highback}{azure9}
\CodeHigh{language=latex/table,style/main=highback,style/code=highback}
\NewCodeHighEnv{code}{style/main=gray9,style/code=gray9}
\NewCodeHighEnv{demo}{style/main=gray9,style/code=gray9,demo}

%\CodeHigh{lite}

\CodeHigh{lite}
\setcounter{chapter}{1}

\begin{document}

\chapter{Basic Interfaces}

\section{Old and New Interfaces}

With \verb!tabularray! package, you can change the styles of tables via old interfaces or new interfaces.

The old interfaces consist of some table commands inside the table contents.
Same as \verb!tabular! and \verb!array! environments,
all table commands \textcolor{red3}{must} be put at the beginning ot the cell text.
Also, new table commands \textcolor{red3}{must} be defined with \verb!\NewTableCommand!.

The new interfaces consist of some options inside the mandatory argument,
hence totally separating the styles and the contents of tables.

\begin{newtblr}[
  caption = {Old Interfaces and New Interfaces},
  label = {key:interface},
]{verb}
  Old Interfaces                                 & New Interfaces          \\
  \verb!\SetHlines!                              & \K{hlines}              \\
  \verb!\SetHline!, \verb!\hline!, \verb!\cline! & \K{hline}, \K{rowspec}  \\
  \verb!\SetVlines!                              & \K{vlines}              \\
  \verb!\SetVline!, \verb!\vline!, \verb!\rline! & \K{vline}, \K{colspec}  \\
  \verb!\SetCells!                               & \K{cells}               \\
  \verb!\SetCell!                                & \K{cell}                \\
  \verb!\SetRows!                                & \K{rows}                \\
  \verb!\SetRow!                                 & \K{row}, \K{rowspec}    \\
  \verb!\SetColumns!                             & \K{columns}             \\
  \verb!\SetColumn!                              & \K{column}, \K{colspec} \\
\end{newtblr}

\section{Hlines and Vlines}

All available keys for hlines and vlines are described in Table \ref{key:hvline}.

\begin{spectblr}[
  caption = {Keys for Hlines and Vlines},
  label = {key:hvline},
  remark{Note} = {In most cases, you can omit the underlined key names and write only their values.}
]{}
  Key & Description and Values & Initial Value \\
  \underline{\K{dash}} & dash style: \V{solid}, \V{dashed} or \V{dotted} & \V{solid} \\
  \K{text}             & replace hline/vline with text (like \V{!} specifier in \K{colspec}) & None \\
  \underline{\K{wd}}   & rule width dimension & None \\
  \underline{\K{fg}}   & rule color name & None \\
\end{spectblr}

\subsection{Hlines and Vlines in New Interfaces}

Options \verb!hlines! and \verb!vlines! are for setting all hlines and vlines, respectively.
With empty value, all hlines/vlines will be solid.

\begin{demohigh}
\begin{tblr}{hlines,vlines}
 Alpha   & Beta  & Gamma   & Delta   \\
 Epsilon & Zeta  & Eta     & Theta   \\
 Iota    & Kappa & Lambda  & Mu      \\
 Nu      & Xi    & Omicron & Pi      \\
 Rho     & Sigma & Tau     & Upsilon \\
 Phi     & Chi   & Psi     & Omega   \\
\end{tblr}
\end{demohigh}

With values inside one pair of braces, all hlines/vlines will be styled.

\begin{demohigh}
\begin{tblr}{
 hlines = {1pt,solid},
 vlines = {red3,dashed},
}
 Alpha   & Beta  & Gamma   & Delta   \\
 Epsilon & Zeta  & Eta     & Theta   \\
 Iota    & Kappa & Lambda  & Mu      \\
 Nu      & Xi    & Omicron & Pi      \\
 Rho     & Sigma & Tau     & Upsilon \\
 Phi     & Chi   & Psi     & Omega   \\
\end{tblr}
\end{demohigh}

Another pair of braces before will select segments in all hlines/vlines.

\begin{demohigh}
\begin{tblr}{
 vlines = {1,3,5}{dashed},
 vlines = {2,4,6}{solid},
}
 Alpha   & Beta  & Gamma   & Delta   \\
 Epsilon & Zeta  & Eta     & Theta   \\
 Iota    & Kappa & Lambda  & Mu      \\
 Nu      & Xi    & Omicron & Pi      \\
 Rho     & Sigma & Tau     & Upsilon \\
 Phi     & Chi   & Psi     & Omega   \\
\end{tblr}
\end{demohigh}

The above example can be simplified with \verb!odd! and \verb!even! values.
(More child selectors can be defined with \verb!\NewChildSelector! command.
Advanced users could read the source code for this.)

\begin{demohigh}
\begin{tblr}{
 vlines = {odd}{dashed},
 vlines = {even}{solid},
}
 Alpha   & Beta  & Gamma   & Delta   \\
 Epsilon & Zeta  & Eta     & Theta   \\
 Iota    & Kappa & Lambda  & Mu      \\
 Nu      & Xi    & Omicron & Pi      \\
 Rho     & Sigma & Tau     & Upsilon \\
 Phi     & Chi   & Psi     & Omega   \\
\end{tblr}
\end{demohigh}

Another pair of braces before will draw more hlines/vlines (in which \verb!-! stands for all line segments).

\begin{demohigh}
\begin{tblr}{
 hlines = {1}{-}{dashed},
 hlines = {2}{-}{solid},
}
 Alpha   & Beta  & Gamma   & Delta   \\
 Epsilon & Zeta  & Eta     & Theta   \\
 Iota    & Kappa & Lambda  & Mu      \\
 Nu      & Xi    & Omicron & Pi      \\
 Rho     & Sigma & Tau     & Upsilon \\
 Phi     & Chi   & Psi     & Omega   \\
\end{tblr}
\end{demohigh}

Options \verb!hline{i}! and \verb!vline{j}! are for setting some hlines and vlines, respectively.

\begin{demohigh}
\begin{tblr}{
 hline{1,7} = {1pt,solid},
 hline{3-5} = {blue3,dashed},
 vline{1,5} = {3-4}{dotted},
}
 Alpha   & Beta  & Gamma   & Delta   \\
 Epsilon & Zeta  & Eta     & Theta   \\
 Iota    & Kappa & Lambda  & Mu      \\
 Nu      & Xi    & Omicron & Pi      \\
 Rho     & Sigma & Tau     & Upsilon \\
 Phi     & Chi   & Psi     & Omega   \\
\end{tblr}
\end{demohigh}

Now we show the usage of \verb!text! key by the following example%
\footnote{Code from \url{https://tex.stackexchange.com/questions/603023/tabularray-and-tabularx-column-separator}.}:

\begin{demohigh}
\begin{tblr}{
  vlines, hlines,
  colspec = {lX[c]X[c]X[c]X[c]},
  vline{2} = {1}{text=\clap{:}},
  vline{3} = {1}{text=\clap{\ch{+}}},
  vline{4} = {1}{text=\clap{\ch{->}}},
  vline{5} = {1}{text=\clap{\ch{+}}},
}
  Equation & \ch{CH4} & \ch{2 O2} & \ch{CO2} & \ch{2 H2O} \\
  Initial  & $n_1$    & $n_2$     & 0        & 0 \\
  Final    & $n_1-x$  & $n_2-2x$  & $x$      & $2x$ \\
\end{tblr}
\end{demohigh}

You need to load \verb!chemmacros! package for the \verb!\ch! command.

\subsection{Hlines and Vlines in Old Interfaces}

The \verb!\hline! command has an optional argument which accepts key-value options.
The available keys are described in Table \ref{key:hvline}.

\begin{demohigh}
\begin{tblr}{llll}
\hline
 Alpha   & Beta  & Gamma  & Delta \\
\hline[dashed]
 Epsilon & Zeta  & Eta    & Theta \\
\hline[dotted]
 Iota    & Kappa & Lambda & Mu    \\
\hline[2pt,blue5]
\end{tblr}
\end{demohigh}

The \verb!\cline! command also has an optional argument which is the same as \verb!\hline!.

\begin{demohigh}
\begin{tblr}{llll}
\cline{1-4}
 Alpha   & Beta  & Gamma  & Delta \\
\cline[dashed]{1,3}
 Epsilon & Zeta  & Eta    & Theta \\
\cline[dashed]{2,4}
 Iota    & Kappa & Lambda & Mu    \\
\cline[2pt,blue5]{-}
\end{tblr}
\end{demohigh}

You can use child selectors in the mandatory argument of \verb!\cline!.

\begin{demohigh}
\begin{tblr}{llll}
\cline{1-4}
 Alpha   & Beta  & Gamma  & Delta \\
\cline[dashed]{odd}
 Epsilon & Zeta  & Eta    & Theta \\
\cline[dashed]{even}
 Iota    & Kappa & Lambda & Mu    \\
\cline[2pt,blue5]{-}
\end{tblr}
\end{demohigh}

\section{Cells and Spancells}

All available keys for cells are described in Table \ref{key:cell} and Table \ref{key:cellspan}.
\nopagebreak
\begin{spectblr}[
  caption = {Keys for the Content of Cells},
  label = {key:cell},
  remark{Note} = {In most cases, you can omit the underlined key names and write only their values.}
]{}
  Key & Description and Values & Initial Value \\
  \underline{\K{halign}}
    & horizontal alignment: \V{l} (left), \V{c} (center), or \V{r} (right)
    & \V{l} \\
  \underline{\K{valign}}
    & vertical alignment: \V{t} (top), \V{m} (middle), \V{b} (bottom),
      \V{h} (head) or \V{f} (foot)
    & \V{t} \\
  \underline{\K{wd}} & width dimension & None \\
  \underline{\K{bg}} & background color name & None \\
  \K{fg}    & foreground color name & None \\
  \K{font}  & font commands & None \\
  \K{preto} & prepend text to the cell & None \\
  \K{appto} & append text to the cell & None \\
  \K{cmd}   & execute command for the cell text & None \\
\end{spectblr}
\vspace{-2em}
\begin{spectblr}[
  caption = {Keys for Multispan of Cells},
  label = {key:cellspan},
]{}
  Key & Description and Values & Initial Value \\
  \K{r} & number of rows the cell spans    & 1 \\
  \K{c} & number of columns the cell spans & 1 \\
\end{spectblr}

\subsection{Cells and Spancells in New Interfaces}

Option \verb!cells! is for setting all cells.
\nopagebreak
\begin{demohigh}
\begin{tblr}{hlines={white},cells={c,blue7}}
 Alpha   & Beta  & Gamma   & Delta   \\
 Epsilon & Zeta  & Eta     & Theta   \\
 Iota    & Kappa & Lambda  & Mu      \\
 Nu      & Xi    & Omicron & Pi      \\
\end{tblr}
\end{demohigh}

Option \verb!cell{i}{j}! is for setting some cells.

\begin{demohigh}
\begin{tblr}{
 hlines = {white},
 vlines = {white},
 cell{1,6}{odd} = {teal7},
 cell{1,6}{even} = {green7},
 cell{2,4}{1,4} = {red7},
 cell{3,5}{1,4} = {purple7},
 cell{2}{2} = {r=4,c=2}{c,azure7},
}
 Alpha   & Beta  & Gamma   & Delta   \\
 Epsilon & Zeta  & Eta     & Theta   \\
 Iota    & Kappa & Lambda  & Mu      \\
 Nu      & Xi    & Omicron & Pi      \\
 Rho     & Sigma & Tau     & Upsilon \\
 Phi     & Chi   & Psi     & Omega   \\
\end{tblr}
\end{demohigh}

\subsection{Cells and Spancells in Old Interfaces}

The \verb!\SetCell! command has a mandatory argument for setting the styles of current cell.
The available keys are described in Table \ref{key:cell}.

\begin{demohigh}
\begin{tblr}{llll}
\hline[1pt]
 Alpha   & \SetCell{bg=teal2,fg=white} Beta & Gamma & Delta \\
\hline
 Epsilon & Zeta & \SetCell{r,font=\scshape} Eta & Theta \\
\hline
 Iota    & Kappa & Lambda & Mu    \\
\hline[1pt]
\end{tblr}
\end{demohigh}

The \verb!\SetCell! command also has an optional argument for setting the multispan of current cell.
The available keys are described in Table \ref{key:cellspan}.

\begin{demohigh}
\begin{tblr}{|X|X|X|X|X|X|}
\hline
 Alpha & Beta & Gamma & Delta & Epsilon & Zeta \\
\hline
 \SetCell[c=2]{c} Eta & 2-2
              & \SetCell[c=2]{c} Iota & 2-4
                              & \SetCell[c=2]{c} Lambda  & 2-6 \\
\hline
 \SetCell[c=3]{l} Nu & 3-2 & 3-3
                      & \SetCell[c=3]{l} Pi & 3-5 & 3-6   \\
\hline
 \SetCell[c=6]{r} Tau & 4-2 & 4-3 & 4-4 & 4-5 & 4-6 \\
\hline
\end{tblr}
\end{demohigh}

\section{Rows and Columns}

All available keys for rows and columns are described in Table \ref{key:row} and Table \ref{key:column}.

\begin{spectblr}[
  caption = {Keys for Rows},
  label = {key:row},
  remark{Note} = {In most cases, you can omit the underlined key names and write only their values.}
]{}
  Key & Description and Values & Initial Value \\
  \underline{\K{halign}}
    & horizontal alignment: \V{l} (left), \V{c} (center), or \V{r} (right)
    & \V{l} \\
  \underline{\K{valign}}
    & vertical alignment: \V{t} (top), \V{m} (middle), \V{b} (bottom),
      \V{h} (head) or \V{f} (foot)
    & \V{t} \\
  \underline{\K{ht}} & height dimension & None \\
  \underline{\K{bg}} & background color name & None \\
  \K{fg} & foreground color name & None \\
  \K{font} & font commands & None \\
  \K{abovesep} & set vertical space above the row & \V{2pt} \\
  \K{abovesep+} & increase vertical space above the row & None \\
  \K{belowsep} & set vertical space below the row & \V{2pt} \\
  \K{belowsep+} & increase vertical space below the row & None \\
  \K{rowsep} & set vertical space above and below the row & \V{2pt} \\
  \K{rowsep+} & increase vertical space above and below the row & None \\
  \K{preto} & prepend text to every cell (like \V{>} specifier in \K{rowspec}) & None \\
  \K{appto} & append text to every cell (like \V{<} specifier in \K{rowspec}) & None \\
  \K{cmd}   & execute command for every cell text & None \\
\end{spectblr}
\vspace{-2em}
\begin{spectblr}[
  caption = {Keys for Columns},
  label = {key:column},
  remark{Note} = {In most cases, you can omit the underlined key names and write only their values.}
]{}
  Key & Description and Values & Initial Value \\
  \underline{\K{halign}}
    & horizontal alignment: \V{l} (left), \V{c} (center), or \V{r} (right)
    & \V{l} \\
  \underline{\K{valign}}
    & vertical alignment: \V{t} (top), \V{m} (middle), \V{b} (bottom),
      \V{h} (head) or \V{f} (foot)
    & \V{t} \\
  \underline{\K{wd}} & width dimension & None \\
  \underline{\K{co}} & coefficient for the extendable column (\V{X} column) & None \\
  \underline{\K{bg}} & background color name & None \\
  \K{fg} & foreground color name & None \\
  \K{font} & font commands & None \\
  \K{leftsep} & set horizontal space to the left of the column & \V{6pt} \\
  \K{leftsep+} & increase horizontal space to the left of the column & None \\
  \K{rightsep} & set horizontal space to the right of the column & \V{6pt} \\
  \K{rightsep+} & increase horizontal space to the right of the column & None \\
  \K{colsep} & set horizontal space to both sides of the column & \V{6pt} \\
  \K{colsep+} & increase horizontal space to both sides of the column & None \\
  \K{preto} & prepend text to every cell (like \V{>} specifier in \K{colspec}) & None \\
  \K{appto} & append text to every cell (like \V{<} specifier in \K{colspec}) & None \\
  \K{cmd}   & execute command for every cell text & None \\
\end{spectblr}

\subsection{Rows and Columns in New Interfaces}

Options \verb!rows! and \verb!columns! are for setting all rows and columns, respectively.
\nopagebreak
\begin{demohigh}
\begin{tblr}{
 hlines,
 vlines,
 rows = {7mm},
 columns = {15mm,c},
}
 Alpha   & Beta  & Gamma   & Delta \\
 Epsilon & Zeta  & Eta     & Theta \\
 Iota    & Kappa & Lambda  & Mu    \\
\end{tblr}
\end{demohigh}

Options \verb!row{i}! and \verb!column{j}! are for setting some rows and columns, respectively.

\begin{demohigh}
\begin{tblr}{
 hlines = {1pt,white},
 row{odd} = {blue7},
 row{even} = {azure7},
 column{1} = {purple7,c},
}
 Alpha   & Beta  & Gamma   & Delta   \\
 Epsilon & Zeta  & Eta     & Theta   \\
 Iota    & Kappa & Lambda  & Mu      \\
 Nu      & Xi    & Omicron & Pi      \\
 Rho     & Sigma & Tau     & Upsilon \\
 Phi     & Chi   & Psi     & Omega   \\
\end{tblr}
\end{demohigh}

The following example demonstrates the usages of \verb!bg!, \verb!fg! and \verb!font! keys.
\nopagebreak
\begin{demohigh}
\begin{tblr}{
 row{odd} = {bg=azure8},
 row{1}   = {bg=azure3, fg=white, font=\sffamily},
}
 Alpha & Beta    & Gamma \\
 Delta & Epsilon & Zeta  \\
 Eta   & Theta   & Iota  \\
 Kappa & Lambda  & Mu    \\
 Nu Xi Omicron & Pi Rho Sigma & Tau Upsilon Phi \\
\end{tblr}
\end{demohigh}

The following example demonstrates the usages of
\verb!abovesep!, \verb!belowsep!, \verb!leftsep!, \verb!rightsep! keys.
\nopagebreak
\begin{demohigh}
\begin{tblr}{
 hlines,
 vlines,
 rows = {abovesep=1pt,belowsep=5pt},
 columns = {leftsep=1pt,rightsep=5pt},
}
 Alpha   & Beta  & Gamma  & Delta \\
 Epsilon & Zeta  & Eta    & Theta \\
 Iota    & Kappa & Lambda & Mu    \\
\end{tblr}
\end{demohigh}

The following example shows that we can replace \verb!\\[dimen]! with \verb!belowsep+! key.

\begin{demohigh}
\begin{tblr}{
 hlines, row{2} = {belowsep+=5pt},
}
 Alpha   & Beta  & Gamma  & Delta \\
 Epsilon & Zeta  & Eta    & Theta \\
 Iota    & Kappa & Lambda & Mu    \\
\end{tblr}
\end{demohigh}

\subsection{Rows and Columns in Old Interfaces}

The \verb!\SetRow! command has a mandatory argument for setting the styles of current row.
The available keys are described in Table \ref{key:row}.

\begin{demohigh}
\begin{tblr}{llll}
\hline[1pt]
 \SetRow{azure8} Alpha & Beta & Gamma & Delta \\
\hline
 \SetRow{blue8,c} Epsilon & Zeta & Eta & Theta \\
\hline
 \SetRow{violet8} Iota & Kappa & Lambda & Mu \\
\hline[1pt]
\end{tblr}
\end{demohigh}

\section{Colspec and Rowspec}

Options \verb!colspec!/\verb!rowspec! are for setting column/row specifications
with column/row type specifiers.

\subsection{Colspec and Width}

Option \verb!width! are for setting the width of the table with extendable columns.
The following example demonstrates the usage of \verb!width! option.
\nopagebreak
\begin{demohigh}
\begin{tblr}{width=0.8\textwidth, colspec={|l|X[2]|X[3]|X[-1]|}}
 Alpha   & Beta  & Gamma  & Delta \\
 Epsilon & Zeta  & Eta    & Theta \\
 Iota    & Kappa & Lambda & Mu    \\
\end{tblr}
\end{demohigh}

\subsection{Column Types}

The \verb!tabularray! package has only one type of primitive column: the \verb!Q! column.
Other types of columns are defined as \verb!Q! columns with some keys.

\begin{codehigh}
\NewColumnType{l}{Q[l]}
\NewColumnType{c}{Q[c]}
\NewColumnType{r}{Q[r]}
\NewColumnType{t}[1]{Q[t,wd=#1]}
\NewColumnType{m}[1]{Q[m,wd=#1]}
\NewColumnType{b}[1]{Q[b,wd=#1]}
\NewColumnType{h}[1]{Q[h,wd=#1]}
\NewColumnType{f}[1]{Q[f,wd=#1]}
\NewColumnType{X}[1][]{Q[co=1,#1]}
\end{codehigh}

\begin{demohigh}
\begin{tblr}{|t{15mm}|m{15mm}|b{20mm}|}
 Alpha   & Beta  & {Gamma\\Gamma} \\
 Epsilon & Zeta  & {Eta\\Eta} \\
 Iota    & Kappa & {Lambda\\Lambda} \\
\end{tblr}
\end{demohigh}

Any new column type must be defined with \verb!\NewColumnType! command.
It can have an optional argument when it's defined.

\subsection{Row Types}

The \verb!tabularray! package has only one type of primitive row: the \verb!Q! row.
Other types of rows are defined as \verb!Q! rows with some keys.

\begin{codehigh}
\NewRowType{l}{Q[l]}
\NewRowType{c}{Q[c]}
\NewRowType{r}{Q[r]}
\NewRowType{t}[1]{Q[t,ht=#1]}
\NewRowType{m}[1]{Q[m,ht=#1]}
\NewRowType{b}[1]{Q[b,ht=#1]}
\NewRowType{h}[1]{Q[h,ht=#1]}
\NewRowType{f}[1]{Q[f,ht=#1]}
\end{codehigh}

\begin{demohigh}
\begin{tblr}{rowspec={|t{12mm}|m{10mm}|b{10mm}|}}
 Alpha   & Beta  & {Gamma\\Gamma} \\
 Epsilon & Zeta  & {Eta\\Eta} \\
 Iota    & Kappa & {Lambda\\Lambda} \\
\end{tblr}
\end{demohigh}

Any new row type must be defined with \verb!\NewRowType! command.
It can have an optional argument when it's defined.

\end{document}
