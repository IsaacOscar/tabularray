% -*- coding: utf-8 -*-
% !TEX program = lualatex
\documentclass[oneside]{book}

% -*- coding: utf-8 -*-
% !TEX program = lualatex

\newcommand*{\myversion}{2022D}
\newcommand*{\mylpad}[1]{\ifnum#1<10 0\the#1\else\the#1\fi}

\usepackage[a4paper,margin=2.5cm]{geometry}

\setlength{\parindent}{0pt}
\setlength{\parskip}{4pt plus 1pt minus 1pt}

\usepackage{codehigh} % https://ctan.org/pkg/codehigh
\usepackage{tabularray}
\usepackage{array,multirow,amsmath}
\usepackage{chemmacros,environ}
\usepackage{enumitem}

\usepackage[firstpage=true]{background}
\backgroundsetup{contents={}}

\UseTblrLibrary{amsmath,booktabs,counter,diagbox,functional,siunitx,varwidth}

\usepackage{hyperref}
\hypersetup{
  colorlinks=true,
  urlcolor=blue3,
  linkcolor=blue3,
}

\usepackage{tcolorbox}
\tcbset{sharp corners, boxrule=0.5pt, colback=red9}

\usepackage{float}
%\usepackage{enumerate}

\setcounter{tocdepth}{1}

\newcommand*{\K}[1]{\texttt{#1}}
\newcommand*{\V}[1]{\texttt{#1}}
\newcommand*{\None}{$\times$}

\NewTblrEnviron{newtblr}
\SetTblrOuter[newtblr]{long}
\SetTblrInner[newtblr]{
  hlines = {white}, column{1,2} = {co=1}, colsep = 5pt,
  row{odd} = {brown8}, row{even} = {gray8},
  row{1} = {fg=white, bg=purple2, font=\bfseries\sffamily},
}

\NewTblrEnviron{spectblr}
\SetTblrOuter[spectblr]{long}
\SetTblrInner[spectblr]{
  hlines = {white}, column{2} = {co=1}, colsep = 5pt,
  row{odd} = {brown8}, row{even} = {gray8},
  row{1} = {fg=white, bg=purple2, font=\bfseries\sffamily},
  rowhead = 1,
}

\newcommand{\mywarning}[1]{%
  \begin{tcolorbox}
  The interfaces in this #1 should be seen as
  \textcolor{red3}{\bfseries experimental}
  and are likely to change in future releases, if necessary.
  Don’t use them in important documents.
  \end{tcolorbox}
}

%\renewcommand*{\thefootnote}{*}

\colorlet{highback}{azure9}
\CodeHigh{language=latex/table,style/main=highback,style/code=highback}
\NewCodeHighEnv{code}{style/main=gray9,style/code=gray9}
\NewCodeHighEnv{demo}{style/main=gray9,style/code=gray9,demo}

%\CodeHigh{lite}

\CodeHigh{lite}
\setcounter{chapter}{2}

\begin{document}

\chapter{The New Interfaces}

With \verb!tabularray! package, you can separate style and content totally in tables.

\section{Hlines and Vlines}

All available keys for hlines and vlines are described in Table \ref{key:hvline}.

\begin{spectblr}[
  caption = {Keys for Hlines and Vlines},
  label = {key:hvline},
  remark{Note} = {In most cases, you can omit the underlined key names and write only their values.}
]{}
  Key & Description and Values & Initial Value \\
  \underline{\K{dash}} & dash style: \V{solid}, \V{dashed} or \V{dotted} & \V{solid} \\
  \K{text}             & replace hline/vline with text (like \V{!} specifier in \K{colspec}) & None \\
  \underline{\K{wd}}   & rule width dimension & None \\
  \underline{\K{fg}}   & rule color name & None \\
\end{spectblr}

Options \verb!hlines! and \verb!vlines! are for setting all hlines and vlines, respectively.
With empty value, all hlines/vlines will be solid.

\begin{demohigh}
\begin{tblr}{hlines,vlines}
 Alpha   & Beta  & Gamma   & Delta   \\
 Epsilon & Zeta  & Eta     & Theta   \\
 Iota    & Kappa & Lambda  & Mu      \\
 Nu      & Xi    & Omicron & Pi      \\
 Rho     & Sigma & Tau     & Upsilon \\
 Phi     & Chi   & Psi     & Omega   \\
\end{tblr}
\end{demohigh}

With values inside one pair of braces, all hlines/vlines will be styled.

\begin{demohigh}
\begin{tblr}{
 hlines = {1pt,solid},
 vlines = {red3,dashed},
}
 Alpha   & Beta  & Gamma   & Delta   \\
 Epsilon & Zeta  & Eta     & Theta   \\
 Iota    & Kappa & Lambda  & Mu      \\
 Nu      & Xi    & Omicron & Pi      \\
 Rho     & Sigma & Tau     & Upsilon \\
 Phi     & Chi   & Psi     & Omega   \\
\end{tblr}
\end{demohigh}

Another pair of braces before will select segments in all hlines/vlines.

\begin{demohigh}
\begin{tblr}{
 vlines = {1,3,5}{dashed},
 vlines = {2,4,6}{solid},
}
 Alpha   & Beta  & Gamma   & Delta   \\
 Epsilon & Zeta  & Eta     & Theta   \\
 Iota    & Kappa & Lambda  & Mu      \\
 Nu      & Xi    & Omicron & Pi      \\
 Rho     & Sigma & Tau     & Upsilon \\
 Phi     & Chi   & Psi     & Omega   \\
\end{tblr}
\end{demohigh}

The above example can be simplified with \verb!odd! and \verb!even! values.
(More child selectors can be defined with \verb!\NewChildSelector! command.
Advanced users could read the source code for this.)

\begin{demohigh}
\begin{tblr}{
 vlines = {odd}{dashed},
 vlines = {even}{solid},
}
 Alpha   & Beta  & Gamma   & Delta   \\
 Epsilon & Zeta  & Eta     & Theta   \\
 Iota    & Kappa & Lambda  & Mu      \\
 Nu      & Xi    & Omicron & Pi      \\
 Rho     & Sigma & Tau     & Upsilon \\
 Phi     & Chi   & Psi     & Omega   \\
\end{tblr}
\end{demohigh}

Another pair of braces before will draw more hlines/vlines (in which \verb!-! stands for all line segments).

\begin{demohigh}
\begin{tblr}{
 hlines = {1}{-}{dashed},
 hlines = {2}{-}{solid},
}
 Alpha   & Beta  & Gamma   & Delta   \\
 Epsilon & Zeta  & Eta     & Theta   \\
 Iota    & Kappa & Lambda  & Mu      \\
 Nu      & Xi    & Omicron & Pi      \\
 Rho     & Sigma & Tau     & Upsilon \\
 Phi     & Chi   & Psi     & Omega   \\
\end{tblr}
\end{demohigh}

Options \verb!hline{i}! and \verb!vline{j}! are for setting some hlines and vlines, respectively.

\begin{demohigh}
\begin{tblr}{
 hline{1,7} = {1pt,solid},
 hline{3-5} = {blue3,dashed},
 vline{1,5} = {3-4}{dotted},
}
 Alpha   & Beta  & Gamma   & Delta   \\
 Epsilon & Zeta  & Eta     & Theta   \\
 Iota    & Kappa & Lambda  & Mu      \\
 Nu      & Xi    & Omicron & Pi      \\
 Rho     & Sigma & Tau     & Upsilon \\
 Phi     & Chi   & Psi     & Omega   \\
\end{tblr}
\end{demohigh}

At last we show the usage of \verb!text! key by the following example%
\footnote{Code from \url{https://tex.stackexchange.com/questions/603023/tabularray-and-tabularx-column-separator}.}:

\begin{demohigh}
\begin{tblr}{
  vlines, hlines,
  colspec = {lX[c]X[c]X[c]X[c]},
  vline{2} = {1}{text=\clap{:}},
  vline{3} = {1}{text=\clap{\ch{+}}},
  vline{4} = {1}{text=\clap{\ch{->}}},
  vline{5} = {1}{text=\clap{\ch{+}}},
}
  Equation & \ch{CH4} & \ch{2 O2} & \ch{CO2} & \ch{2 H2O} \\
  Initial  & $n_1$    & $n_2$     & 0        & 0 \\
  Final    & $n_1-x$  & $n_2-2x$  & $x$      & $2x$ \\
\end{tblr}
\end{demohigh}

You need to load \verb!chemmacros! package for the \verb!\ch! command.

\section{Cells and Spancells}

All available keys for cells are described in Table \ref{key:cell} and Table \ref{key:cellspan}.
\nopagebreak
\begin{spectblr}[
  caption = {Keys for the Content of Cells},
  label = {key:cell},
  remark{Note} = {In most cases, you can omit the underlined key names and write only their values.}
]{}
  Key & Description and Values & Initial Value \\
  \underline{\K{halign}}
    & horizontal alignment: \V{l} (left), \V{c} (center), or \V{r} (right)
    & \V{l} \\
  \underline{\K{valign}}
    & vertical alignment: \V{t} (top), \V{m} (middle), \V{b} (bottom),
      \V{h} (head) or \V{f} (foot)
    & \V{t} \\
  \underline{\K{wd}} & width dimension & None \\
  \underline{\K{bg}} & background color name & None \\
  \K{fg}    & foreground color name & None \\
  \K{font}  & font commands & None \\
  \K{preto} & prepend text to the cell & None \\
  \K{appto} & append text to the cell & None \\
  \K{cmd}   & execute command for the cell text & None \\
\end{spectblr}
\vspace{-2em}
\begin{spectblr}[
  caption = {Keys for Multispan of Cells},
  label = {key:cellspan},
]{}
  Key & Description and Values & Initial Value \\
  \K{r} & number of rows the cell spans    & 1 \\
  \K{c} & number of columns the cell spans & 1 \\
\end{spectblr}

Option \verb!cells! is for setting all cells.
\nopagebreak
\begin{demohigh}
\begin{tblr}{hlines={white},cells={c,blue7}}
 Alpha   & Beta  & Gamma   & Delta   \\
 Epsilon & Zeta  & Eta     & Theta   \\
 Iota    & Kappa & Lambda  & Mu      \\
 Nu      & Xi    & Omicron & Pi      \\
\end{tblr}
\end{demohigh}

Option \verb!cell{i}{j}! is for setting some cells.

\begin{demohigh}
\begin{tblr}{
 hlines = {white},
 vlines = {white},
 cell{1,6}{odd} = {teal7},
 cell{1,6}{even} = {green7},
 cell{2,4}{1,4} = {red7},
 cell{3,5}{1,4} = {purple7},
 cell{2}{2} = {r=4,c=2}{c,azure7},
}
 Alpha   & Beta  & Gamma   & Delta   \\
 Epsilon & Zeta  & Eta     & Theta   \\
 Iota    & Kappa & Lambda  & Mu      \\
 Nu      & Xi    & Omicron & Pi      \\
 Rho     & Sigma & Tau     & Upsilon \\
 Phi     & Chi   & Psi     & Omega   \\
\end{tblr}
\end{demohigh}

\section{Rows and Columns}

All available keys for rows and columns are described in Table \ref{key:row} and Table \ref{key:column}.

\begin{spectblr}[
  caption = {Keys for Rows},
  label = {key:row},
  remark{Note} = {In most cases, you can omit the underlined key names and write only their values.}
]{}
  Key & Description and Values & Initial Value \\
  \underline{\K{halign}}
    & horizontal alignment: \V{l} (left), \V{c} (center), or \V{r} (right)
    & \V{l} \\
  \underline{\K{valign}}
    & vertical alignment: \V{t} (top), \V{m} (middle), \V{b} (bottom),
      \V{h} (head) or \V{f} (foot)
    & \V{t} \\
  \underline{\K{ht}} & height dimension & None \\
  \underline{\K{bg}} & background color name & None \\
  \K{fg} & foreground color name & None \\
  \K{font} & font commands & None \\
  \K{abovesep} & set vertical space above the row & \V{2pt} \\
  \K{abovesep+} & increase vertical space above the row & None \\
  \K{belowsep} & set vertical space below the row & \V{2pt} \\
  \K{belowsep+} & increase vertical space below the row & None \\
  \K{rowsep} & set vertical space above and below the row & \V{2pt} \\
  \K{rowsep+} & increase vertical space above and below the row & None \\
  \K{preto} & prepend text to every cell (like \V{>} specifier in \K{rowspec}) & None \\
  \K{appto} & append text to every cell (like \V{<} specifier in \K{rowspec}) & None \\
  \K{cmd}   & execute command for every cell text & None \\
\end{spectblr}
\vspace{-2em}
\begin{spectblr}[
  caption = {Keys for Columns},
  label = {key:column},
  remark{Note} = {In most cases, you can omit the underlined key names and write only their values.}
]{}
  Key & Description and Values & Initial Value \\
  \underline{\K{halign}}
    & horizontal alignment: \V{l} (left), \V{c} (center), or \V{r} (right)
    & \V{l} \\
  \underline{\K{valign}}
    & vertical alignment: \V{t} (top), \V{m} (middle), \V{b} (bottom),
      \V{h} (head) or \V{f} (foot)
    & \V{t} \\
  \underline{\K{wd}} & width dimension & None \\
  \underline{\K{co}} & coefficient for the extendable column (\V{X} column) & None \\
  \underline{\K{bg}} & background color name & None \\
  \K{fg} & foreground color name & None \\
  \K{font} & font commands & None \\
  \K{leftsep} & set horizontal space to the left of the column & \V{6pt} \\
  \K{leftsep+} & increase horizontal space to the left of the column & None \\
  \K{rightsep} & set horizontal space to the right of the column & \V{6pt} \\
  \K{rightsep+} & increase horizontal space to the right of the column & None \\
  \K{colsep} & set horizontal space to both sides of the column & \V{6pt} \\
  \K{colsep+} & increase horizontal space to both sides of the column & None \\
  \K{preto} & prepend text to every cell (like \V{>} specifier in \K{colspec}) & None \\
  \K{appto} & append text to every cell (like \V{<} specifier in \K{colspec}) & None \\
  \K{cmd}   & execute command for every cell text & None \\
\end{spectblr}

Options \verb!rows! and \verb!columns! are for setting all rows and columns, respectively.
\nopagebreak
\begin{demohigh}
\begin{tblr}{
 hlines,
 vlines,
 rows = {7mm},
 columns = {15mm,c},
}
 Alpha   & Beta  & Gamma   & Delta \\
 Epsilon & Zeta  & Eta     & Theta \\
 Iota    & Kappa & Lambda  & Mu    \\
\end{tblr}
\end{demohigh}

Options \verb!row{i}! and \verb!column{j}! are for setting some rows and columns, respectively.

\begin{demohigh}
\begin{tblr}{
 hlines = {1pt,white},
 row{odd} = {blue7},
 row{even} = {azure7},
 column{1} = {purple7,c},
}
 Alpha   & Beta  & Gamma   & Delta   \\
 Epsilon & Zeta  & Eta     & Theta   \\
 Iota    & Kappa & Lambda  & Mu      \\
 Nu      & Xi    & Omicron & Pi      \\
 Rho     & Sigma & Tau     & Upsilon \\
 Phi     & Chi   & Psi     & Omega   \\
\end{tblr}
\end{demohigh}

The following example demonstrates the usages of \verb!bg!, \verb!fg! and \verb!font! keys.
\nopagebreak
\begin{demohigh}
\begin{tblr}{
 row{odd} = {bg=azure8},
 row{1}   = {bg=azure3, fg=white, font=\sffamily},
}
 Alpha & Beta    & Gamma \\
 Delta & Epsilon & Zeta  \\
 Eta   & Theta   & Iota  \\
 Kappa & Lambda  & Mu    \\
 Nu Xi Omikron & Pi Rho Sigma & Tau Upsilon Phi \\
\end{tblr}
\end{demohigh}

The following example demonstrates the usages of
\verb!abovesep!, \verb!belowsep!, \verb!leftsep!, \verb!rightsep! keys.
\nopagebreak
\begin{demohigh}
\begin{tblr}{
 hlines,
 vlines,
 rows = {abovesep=1pt,belowsep=5pt},
 columns = {leftsep=1pt,rightsep=5pt},
}
 Alpha   & Beta  & Gamma  & Delta \\
 Epsilon & Zeta  & Eta    & Theta \\
 Iota    & Kappa & Lambda & Mu    \\
\end{tblr}
\end{demohigh}

The following example shows that we can replace \verb!\\[dimen]! with \verb!belowsep+! key.

\begin{demohigh}
\begin{tblr}{
 hlines, row{2} = {belowsep+=5pt},
}
 Alpha   & Beta  & Gamma  & Delta \\
 Epsilon & Zeta  & Eta    & Theta \\
 Iota    & Kappa & Lambda & Mu    \\
\end{tblr}
\end{demohigh}

\section{The Whole Table}

All available keys for the whole table are described in Table \ref{key:table}.

\begin{spectblr}[
  caption = {Keys for the Whole Table},
  label = {key:table},
]{}
  Key & Description and Values & Initial Value \\
  \K{rowspec} & set row specifications with row type specifiers & None \\
  \K{colspec} & set column specifications with column type specifiers & None \\
  \K{width} & width of the table with extendable columns & None \\
  \K{rulesep} & space between two hlines or vlines & \V{2pt} \\
  \K{stretch} & stretch ratio for struts added to cell text & \V{1} \\
  \K{abovesep} & set vertical space above every row & \V{2pt} \\
  \K{belowsep} & set vertical space below every row & \V{2pt} \\
  \K{rowsep} & set vertical space above and below every row & \V{2pt} \\
  \K{leftsep} & set horizontal space to the left of every column & \V{6pt} \\
  \K{rightsep} & set horizontal space to the right of every column & \V{6pt} \\
  \K{colsep} & set horizontal space to both sides of every column & \V{6pt} \\
  \K{hspan} & horizontal span algorithm: \V{default}, \V{even}, or \V{minimal} & \V{default} \\
  \K{vspan} & vertical span algorithm: \V{default} or \V{even} & \V{default} \\
\end{spectblr}

The following example demonstrates the usage of \verb!width! key.
\nopagebreak
\begin{demohigh}
\begin{tblr}{width=0.8\textwidth, colspec={|l|X[2]|X[3]|X[-1]|}}
 Alpha   & Beta  & Gamma  & Delta \\
 Epsilon & Zeta  & Eta    & Theta \\
 Iota    & Kappa & Lambda & Mu    \\
\end{tblr}
\end{demohigh}

The following example shows that we can replace \verb!\doublerulesep! parameter with \verb!rulesep! key.
\nopagebreak
\begin{demohigh}
\begin{tblr}{
 colspec={||llll||},rowspec={|QQQ|},rulesep=4pt,
}
 Alpha   & Beta  & Gamma  & Delta \\
 Epsilon & Zeta  & Eta    & Theta \\
 Iota    & Kappa & Lambda & Mu    \\
\end{tblr}
\end{demohigh}

The following example shows that we can replace \verb!\arraystretch! parameter with \verb!stretch! key.

\begin{demohigh}
\begin{tblr}{hlines,stretch=1.5}
 Alpha   & Beta  & Gamma  & Delta \\
 Epsilon & Zeta  & Eta    & Theta \\
 Iota    & Kappa & Lambda & Mu    \\
\end{tblr}
\end{demohigh}

The following example uses \verb!rowsep! and \verb!colsep! keys to set padding for all rows and columns.
\nopagebreak
\begin{demohigh}
\SetTblrInner{rowsep=2pt,colsep=2pt}
\begin{tblr}{hlines,vlines}
 Alpha   & Beta  & Gamma  & Delta \\
 Epsilon & Zeta  & Eta    & Theta \\
 Iota    & Kappa & Lambda & Mu    \\
\end{tblr}
\end{demohigh}

With \verb!hspan=default! or \verb!hspan=even!,
\verb!tabularray! package will compute column widths from span widths.
But with \verb!hspan=minimal!, it will compute span widths from column widths.
The following examples show the results from different \verb!hspan! values.

\begin{demohigh}
\SetTblrInner{hlines, vlines, hspan=default}
\begin{tblr}{cell{2}{1}={c=2}{l},cell{3}{1}={c=3}{l},cell{4}{2}={c=2}{l}}
 111 111 & 222 222 & 333 333 \\
 12 Multi Columns Multi Columns 12 & & 333 \\
 13 Multi Columns Multi Columns Multi Columns 13 & & \\
 111 & 23 Multi Columns Multi Columns 23 & \\
\end{tblr}
\end{demohigh}

\begin{demohigh}
\SetTblrInner{hlines, vlines, hspan=even}
\begin{tblr}{cell{2}{1}={c=2}{l},cell{3}{1}={c=3}{l},cell{4}{2}={c=2}{l}}
 111 111 & 222 222 & 333 333 \\
 12 Multi Columns Multi Columns 12 & & 333 \\
 13 Multi Columns Multi Columns Multi Columns 13 & & \\
 111 & 23 Multi Columns Multi Columns 23 & \\
\end{tblr}
\end{demohigh}

\begin{demohigh}
\SetTblrInner{hlines, vlines, hspan=minimal}
\begin{tblr}{cell{2}{1}={c=2}{l},cell{3}{1}={c=3}{l},cell{4}{2}={c=2}{l}}
 111 111 & 222 222 & 333 333 \\
 12 Multi Columns Multi Columns 12 & & 333 \\
 13 Multi Columns Multi Columns Multi Columns 13 & & \\
 111 & 23 Multi Columns Multi Columns 23 & \\
\end{tblr}
\end{demohigh}

The following examples show the results from different \verb!vspan! values.
\nopagebreak
\begin{demohigh}
\SetTblrInner{hlines, vlines, vspan=default}
\begin{tblr}{column{2}={3.25cm}, cell{2}{2}={r=3}{l}}
  Column1 & Column2 \\
  Row1 & Long text that needs multiple lines.
         Long text that needs multiple lines.
         Long text that needs multiple lines. \\
  Row2 & \\
  Row3 & \\
  Row4 & Short text \\
\end{tblr}
\end{demohigh}

\begin{demohigh}
\SetTblrInner{hlines, vlines, vspan=even}
\begin{tblr}{column{2}={3.25cm}, cell{2}{2}={r=3}{l}}
  Column1 & Column2 \\
  Row1 & Long text that needs multiple lines.
         Long text that needs multiple lines.
         Long text that needs multiple lines. \\
  Row2 & \\
  Row3 & \\
  Row4 & Short text \\
\end{tblr}
\end{demohigh}

\section{Default Specifications}

\verb!Tabularray! package provides \verb!\SetTblrInner! and \verb!\SetTblrOuter! commands
for you to change the default inner and outer specifications of tables.
Inner specifications are all specifications written in the mandatory argument of the \verb!tblr! environment,
while outer specifications are all specifications written in the optional argument of the \verb!tblr! environment.
At this time, most of the outer specifications are used for long tables (see Chapter \ref{chap:long}).

In the below example, the first line draws all hlines and vlines for all tables created afterwards,
while the second line makes all tables created afterwards vertically align at bottom.

\begin{codehigh}
\SetTblrInner{hlines,vlines}
\SetTblrOuter{valign=b}
\end{codehigh}

You can define new \verb!tabularray! environments using \verb!\NewTblrEnviron! command:

\begin{demohigh}
\NewTblrEnviron{mytblr}
\SetTblrInner[mytblr]{hlines,vlines}
\SetTblrOuter[mytblr]{valign=b}
Text \begin{mytblr}{cccc}
 Alpha   & Beta  & Gamma  & Delta \\
 Epsilon & Zeta  & Eta    & Theta \\
 Iota    & Kappa & Lambda & Mu    \\
\end{mytblr} Text
\end{demohigh}

\section{Counters and Lengths}

Counters \verb!rownum!, \verb!colnum!, \verb!rowcount!, \verb!colcount! can be used in cell text:
\nopagebreak
\begin{demohigh}
\begin{tblr}{hlines}
 Cell[\arabic{rownum}][\arabic{colnum}] & Cell[\arabic{rownum}][\arabic{colnum}] &
 Cell[\arabic{rownum}][\arabic{colnum}] & Cell[\arabic{rownum}][\arabic{colnum}] \\
 Row=\arabic{rowcount}, Col=\arabic{colcount} &
 Row=\arabic{rowcount}, Col=\arabic{colcount} &
 Row=\arabic{rowcount}, Col=\arabic{colcount} &
 Row=\arabic{rowcount}, Col=\arabic{colcount} \\
 Cell[\arabic{rownum}][\arabic{colnum}] & Cell[\arabic{rownum}][\arabic{colnum}] &
 Cell[\arabic{rownum}][\arabic{colnum}] & Cell[\arabic{rownum}][\arabic{colnum}] \\
\end{tblr}
\end{demohigh}

Also, lengths \verb!\leftsep!, \verb!\rightsep!, \verb!\abovesep!, \verb!\belowsep! can be used in cell text.

\section{Tracing Tabularray}

To trace internal data behind \verb!tblr! environment, you can use \verb!\SetTblrTracing! command.
For example, \verb!\SetTblrTracing{all}! will turn on all tracings,
and \verb!\SetTblrTracing{none}! will turn off all tracings.
\verb!\SetTblrTracing{+row,+column}! will only tracing row and column data.
All tracing messages will be written to the log files.

\end{document}
