% -*- coding: utf-8 -*-
% !TEX program = lualatex
\documentclass[oneside]{book}

% -*- coding: utf-8 -*-
% !TEX program = lualatex

\newcommand*{\myversion}{2022D}
\newcommand*{\mylpad}[1]{\ifnum#1<10 0\the#1\else\the#1\fi}

\usepackage[a4paper,margin=2.5cm]{geometry}

\setlength{\parindent}{0pt}
\setlength{\parskip}{4pt plus 1pt minus 1pt}

\usepackage{codehigh} % https://ctan.org/pkg/codehigh
\usepackage{tabularray}
\usepackage{array,multirow,amsmath}
\usepackage{chemmacros,environ}
\usepackage{enumitem}

\usepackage[firstpage=true]{background}
\backgroundsetup{contents={}}

\UseTblrLibrary{amsmath,booktabs,counter,diagbox,functional,siunitx,varwidth}

\usepackage{hyperref}
\hypersetup{
  colorlinks=true,
  urlcolor=blue3,
  linkcolor=blue3,
}

\usepackage{tcolorbox}
\tcbset{sharp corners, boxrule=0.5pt, colback=red9}

\usepackage{float}
%\usepackage{enumerate}

\setcounter{tocdepth}{1}

\newcommand*{\K}[1]{\texttt{#1}}
\newcommand*{\V}[1]{\texttt{#1}}
\newcommand*{\None}{$\times$}

\NewTblrEnviron{newtblr}
\SetTblrOuter[newtblr]{long}
\SetTblrInner[newtblr]{
  hlines = {white}, column{1,2} = {co=1}, colsep = 5pt,
  row{odd} = {brown8}, row{even} = {gray8},
  row{1} = {fg=white, bg=purple2, font=\bfseries\sffamily},
}

\NewTblrEnviron{spectblr}
\SetTblrOuter[spectblr]{long}
\SetTblrInner[spectblr]{
  hlines = {white}, column{2} = {co=1}, colsep = 5pt,
  row{odd} = {brown8}, row{even} = {gray8},
  row{1} = {fg=white, bg=purple2, font=\bfseries\sffamily},
  rowhead = 1,
}

\newcommand{\mywarning}[1]{%
  \begin{tcolorbox}
  The interfaces in this #1 should be seen as
  \textcolor{red3}{\bfseries experimental}
  and are likely to change in future releases, if necessary.
  Don’t use them in important documents.
  \end{tcolorbox}
}

%\renewcommand*{\thefootnote}{*}

\colorlet{highback}{azure9}
\CodeHigh{language=latex/table,style/main=highback,style/code=highback}
\NewCodeHighEnv{code}{style/main=gray9,style/code=gray9}
\NewCodeHighEnv{demo}{style/main=gray9,style/code=gray9,demo}

%\CodeHigh{lite}

\CodeHigh{lite}
\setcounter{chapter}{2}

\begin{document}

\chapter{Extra Interfaces}

\section{The Whole Table}

All available keys for the whole table are described in Table \ref{key:table}.

\begin{spectblr}[
  caption = {Keys for the Whole Table},
  label = {key:table},
]{}
  Key & Description and Values & Initial Value \\
  \K{rulesep} & space between two hlines or vlines & \V{2pt} \\
  \K{stretch} & stretch ratio for struts added to cell text & \V{1} \\
  \K{abovesep} & set vertical space above every row & \V{2pt} \\
  \K{belowsep} & set vertical space below every row & \V{2pt} \\
  \K{rowsep} & set vertical space above and below every row & \V{2pt} \\
  \K{leftsep} & set horizontal space to the left of every column & \V{6pt} \\
  \K{rightsep} & set horizontal space to the right of every column & \V{6pt} \\
  \K{colsep} & set horizontal space to both sides of every column & \V{6pt} \\
  \K{hspan} & horizontal span algorithm: \V{default}, \V{even}, or \V{minimal} & \V{default} \\
  \K{vspan} & vertical span algorithm: \V{default} or \V{even} & \V{default} \\
\end{spectblr}

The following example shows that we can replace \verb!\doublerulesep! parameter with \verb!rulesep! key.
\nopagebreak
\begin{demohigh}
\begin{tblr}{
 colspec={||llll||},rowspec={|QQQ|},rulesep=4pt,
}
 Alpha   & Beta  & Gamma  & Delta \\
 Epsilon & Zeta  & Eta    & Theta \\
 Iota    & Kappa & Lambda & Mu    \\
\end{tblr}
\end{demohigh}

The following example shows that we can replace \verb!\arraystretch! parameter with \verb!stretch! key.

\begin{demohigh}
\begin{tblr}{hlines,stretch=1.5}
 Alpha   & Beta  & Gamma  & Delta \\
 Epsilon & Zeta  & Eta    & Theta \\
 Iota    & Kappa & Lambda & Mu    \\
\end{tblr}
\end{demohigh}

The following example uses \verb!rowsep! and \verb!colsep! keys to set padding for all rows and columns.
\nopagebreak
\begin{demohigh}
\SetTblrInner{rowsep=2pt,colsep=2pt}
\begin{tblr}{hlines,vlines}
 Alpha   & Beta  & Gamma  & Delta \\
 Epsilon & Zeta  & Eta    & Theta \\
 Iota    & Kappa & Lambda & Mu    \\
\end{tblr}
\end{demohigh}

With \verb!hspan=default! or \verb!hspan=even!,
\verb!tabularray! package will compute column widths from span widths.
But with \verb!hspan=minimal!, it will compute span widths from column widths.
The following examples show the results from different \verb!hspan! values.

\begin{demohigh}
\SetTblrInner{hlines, vlines, hspan=default}
\begin{tblr}{cell{2}{1}={c=2}{l},cell{3}{1}={c=3}{l},cell{4}{2}={c=2}{l}}
 111 111 & 222 222 & 333 333 \\
 12 Multi Columns Multi Columns 12 & & 333 \\
 13 Multi Columns Multi Columns Multi Columns 13 & & \\
 111 & 23 Multi Columns Multi Columns 23 & \\
\end{tblr}
\end{demohigh}

\begin{demohigh}
\SetTblrInner{hlines, vlines, hspan=even}
\begin{tblr}{cell{2}{1}={c=2}{l},cell{3}{1}={c=3}{l},cell{4}{2}={c=2}{l}}
 111 111 & 222 222 & 333 333 \\
 12 Multi Columns Multi Columns 12 & & 333 \\
 13 Multi Columns Multi Columns Multi Columns 13 & & \\
 111 & 23 Multi Columns Multi Columns 23 & \\
\end{tblr}
\end{demohigh}

\begin{demohigh}
\SetTblrInner{hlines, vlines, hspan=minimal}
\begin{tblr}{cell{2}{1}={c=2}{l},cell{3}{1}={c=3}{l},cell{4}{2}={c=2}{l}}
 111 111 & 222 222 & 333 333 \\
 12 Multi Columns Multi Columns 12 & & 333 \\
 13 Multi Columns Multi Columns Multi Columns 13 & & \\
 111 & 23 Multi Columns Multi Columns 23 & \\
\end{tblr}
\end{demohigh}

The following examples show the results from different \verb!vspan! values.
\nopagebreak
\begin{demohigh}
\SetTblrInner{hlines, vlines, vspan=default}
\begin{tblr}{column{2}={3.25cm}, cell{2}{2}={r=3}{l}}
  Column1 & Column2 \\
  Row1 & Long text that needs multiple lines.
         Long text that needs multiple lines.
         Long text that needs multiple lines. \\
  Row2 & \\
  Row3 & \\
  Row4 & Short text \\
\end{tblr}
\end{demohigh}

\begin{demohigh}
\SetTblrInner{hlines, vlines, vspan=even}
\begin{tblr}{column{2}={3.25cm}, cell{2}{2}={r=3}{l}}
  Column1 & Column2 \\
  Row1 & Long text that needs multiple lines.
         Long text that needs multiple lines.
         Long text that needs multiple lines. \\
  Row2 & \\
  Row3 & \\
  Row4 & Short text \\
\end{tblr}
\end{demohigh}

\section{Default Specifications}

\verb!Tabularray! package provides \verb!\SetTblrInner! and \verb!\SetTblrOuter! commands
for you to change the default inner and outer specifications of tables.
Inner specifications are all specifications written in the mandatory argument of the \verb!tblr! environment,
while outer specifications are all specifications written in the optional argument of the \verb!tblr! environment.
At this time, most of the outer specifications are used for long tables (see Chapter \ref{chap:long}).

In the below example, the first line draws all hlines and vlines for all tables created afterwards,
while the second line makes all tables created afterwards vertically align at bottom.

\begin{codehigh}
\SetTblrInner{hlines,vlines}
\SetTblrOuter{valign=b}
\end{codehigh}

You can define new \verb!tabularray! environments using \verb!\NewTblrEnviron! command:

\begin{demohigh}
\NewTblrEnviron{mytblr}
\SetTblrInner[mytblr]{hlines,vlines}
\SetTblrOuter[mytblr]{valign=b}
Text \begin{mytblr}{cccc}
 Alpha   & Beta  & Gamma  & Delta \\
 Epsilon & Zeta  & Eta    & Theta \\
 Iota    & Kappa & Lambda & Mu    \\
\end{mytblr} Text
\end{demohigh}

\section{New Table Commands}

All commands which change the specifications of tables \textcolor{red3}{must} be defined with \verb!\NewTableCommand!.
The following example demonstrates how to define similar rules as in \verb!booktabs! package:

\begin{codehigh}
\NewTableCommand\toprule{\hline[0.08em]}
\NewTableCommand\midrule{\hline[0.05em]}
\NewTableCommand\bottomrule{\hline[0.08em]}
\begin{tblr}{llll}
\toprule
 Alpha   & Beta  & Gamma   & Delta \\
\midrule
 Epsilon & Zeta  & Eta     & Theta \\
 Iota    & Kappa & Lambda  & Mu    \\
 Nu      & Xi    & Omicron & Pi    \\
\bottomrule
\end{tblr}
\end{codehigh}

\section{Counters and Lengths}

Counters \verb!rownum!, \verb!colnum!, \verb!rowcount!, \verb!colcount! can be used in cell text:
\nopagebreak
\begin{demohigh}
\begin{tblr}{hlines}
 Cell[\arabic{rownum}][\arabic{colnum}] & Cell[\arabic{rownum}][\arabic{colnum}] &
 Cell[\arabic{rownum}][\arabic{colnum}] & Cell[\arabic{rownum}][\arabic{colnum}] \\
 Row=\arabic{rowcount}, Col=\arabic{colcount} &
 Row=\arabic{rowcount}, Col=\arabic{colcount} &
 Row=\arabic{rowcount}, Col=\arabic{colcount} &
 Row=\arabic{rowcount}, Col=\arabic{colcount} \\
 Cell[\arabic{rownum}][\arabic{colnum}] & Cell[\arabic{rownum}][\arabic{colnum}] &
 Cell[\arabic{rownum}][\arabic{colnum}] & Cell[\arabic{rownum}][\arabic{colnum}] \\
\end{tblr}
\end{demohigh}

Also, lengths \verb!\leftsep!, \verb!\rightsep!, \verb!\abovesep!, \verb!\belowsep! can be used in cell text.

\section{Tracing Tabularray}

To trace internal data behind \verb!tblr! environment, you can use \verb!\SetTblrTracing! command.
For example, \verb!\SetTblrTracing{all}! will turn on all tracings,
and \verb!\SetTblrTracing{none}! will turn off all tracings.
\verb!\SetTblrTracing{+row,+column}! will only tracing row and column data.
All tracing messages will be written to the log files.

\end{document}
