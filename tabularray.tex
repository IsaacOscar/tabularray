%  -*- coding: utf-8 -*-
\documentclass[oneside]{book}
\usepackage[a4paper,margin=2.5cm]{geometry}

\usepackage{codehigh} % https://github.com/lvjr/codehigh
%\usepackage{arev}
\usepackage{tabularray}
\usepackage{array,multirow,amsmath}

\usepackage{hyperref}
\hypersetup{
  colorlinks=true,
  urlcolor=blue3,
}
\renewcommand*{\thefootnote}{*}

\newcommand*{\myversion}{2021G}
\newcommand*{\mydate}{Version \myversion\myrepo\ (\the\year-\mylpad\month-\mylpad\day)}
\newcommand*{\myrepo}{{\renewcommand*{\thefootnote}{}%
  \footnote{GitHub repository: \url{https://github.com/lvjr/tabularray}.}}}
\newcommand*{\mylpad}[1]{\ifnum#1<10 0\the#1\else\the#1\fi}

\colorlet{highback}{\ifcase\month\or
  red9\or brown9\or yellow9\or cyan9\or azure9\or blue9\or
  red9\or brown9\or yellow9\or cyan9\or azure9\or blue9\fi
}
\CodeHigh{language=latex/table,style/main=highback,style/code=highback}
\NewCodeHighEnv{code}{style/main=gray9,style/code=gray9}
\NewCodeHighEnv{demo}{style/main=gray9,style/code=gray9,demo}

%\CodeHigh{lite}

\begin{document}

\title{\sffamily\color{red3}Tabularray: Typeset Tabulars and Arrays with \LaTeX3}
\author{Jianrui Lyu (tolvjr@163.com)}
\date{\mydate}
\maketitle

\tableofcontents

\chapter{From Old to New}

\section{Vertical Space}

After loading \verb!tabularrray! package in the preamble,
we can use \verb!tblr! environments to typeset tabulars and arrays.
The name \verb!tblr! is short for \verb!tabularray! or \verb!top-bottom-left-right!.
The following is our first example:

\begin{demo}
\begin{tabular}{lccr}
\hline
 Alpha   & Beta  & Gamma  & Delta \\
\hline
 Epsilon & Zeta  & Eta    & Theta \\
\hline
 Iota    & Kappa & Lambda & Mu    \\
\hline
\end{tabular}
\end{demo}

\begin{demohigh}
\begin{tblr}{lccr}
\hline
 Alpha   & Beta  & Gamma  & Delta \\
\hline
 Epsilon & Zeta  & Eta    & Theta \\
\hline
 Iota    & Kappa & Lambda & Mu    \\
\hline
\end{tblr}
\end{demohigh}

You may notice that there is extra space above and below the table rows with \verb!tblr! envirenment.
This space makes the table look better.
If you don't like it, you could use \verb!\SetTblrDefault! command:

\begin{demohigh}
\SetTblrDefault{rowsep=0pt}
\begin{tblr}{lccr}
\hline
 Alpha   & Beta  & Gamma  & Delta \\
\hline
 Epsilon & Zeta  & Eta    & Theta \\
\hline
 Iota    & Kappa & Lambda & Mu    \\
\hline
\end{tblr}
\end{demohigh} 

But in many cases, this rowsep is useful:

\begin{demo}
$\begin{array}{rrr}
\hline
 \dfrac{2}{3} &  \dfrac{2}{3} &  \dfrac{1}{3} \\
 \dfrac{2}{3} & -\dfrac{1}{3} & -\dfrac{2}{3} \\
 \dfrac{1}{3} & -\dfrac{2}{3} &  \dfrac{2}{3} \\
\hline
\end{array}$
\end{demo}

\begin{demohigh}
$\begin{tblr}{rrr}
\hline
 \dfrac{2}{3} &  \dfrac{2}{3} &  \dfrac{1}{3} \\
 \dfrac{2}{3} & -\dfrac{1}{3} & -\dfrac{2}{3} \\
 \dfrac{1}{3} & -\dfrac{2}{3} &  \dfrac{2}{3} \\
\hline
\end{tblr}$
\end{demohigh}

Note that you can use \verb!tblr! in both text and math modes.

\section{Multiline Cells}

It's quite easy to write multiline cells without fixing the column width in \verb!tblr! environments:
just enclose the cell text with braces and use \verb!\\! to break lines:

\begin{demohigh}
\begin{tblr}{|l|c|r|}
\hline
 Left & {Center \\ Cent \\ C} & {Right \\ R} \\
\hline
 {L \\ Left} & {C \\ Cent \\ Center} & R \\
\hline
\end{tblr}
\end{demohigh} 

\section{Cell Alignment}

From time to time,
you may want to specify the horizontal and vertical alignment of cells at the same time.
\verb!Tabularray! package provides a \verb!Q! column for this
(In fact, \verb!Q! column is the only primitive column,
other columns are defined as \verb!Q! columns with some options):

\begin{demohigh}
\begin{tblr}{|Q[l,t]|Q[c,m]|Q[r,b]|}
\hline
 {Top Baseline \\ Left Left} & Middle Center & {Right Right \\ Bottom Baseline} \\
\hline
\end{tblr}
\end{demohigh} 

Note that you can use more meaningful \verb!t! instead of \verb!p! for top baseline alignment.
For some users who are familiar with word processors,
these \verb!t! and \verb!b! columns are counter-intuitive.
In \verb!tabularray! package, there are another two column types \verb!h! and \verb!f!,
which will align cell text at row head and row foot, respectively:

\begin{demohigh}
\begin{tblr}{Q[h,4em]Q[t,4em]Q[m,4em]Q[b,4em]Q[f,4em]}
\hline
 {row\\head} & {top\\line} & {middle} & {line\\bottom} & {row\\foot} \\
\hline
 {row\\head} & {top\\line} & {11\\22\\mid\\44\\55} & {line\\bottom} & {row\\foot} \\
\hline
\end{tblr}
\end{demohigh}

\section{Multirow Cells}

The above \verb!h! and \verb!f! columns are necessary for multirow cells.
In \verb!tabularray!, the \verb!t! and \verb!b! in the optional argument of \verb!\multirow! commands
will be treated as \verb!h! and \verb!f!, respectively:

\begin{demo}
\begin{tabular}{|l|l|l|l|}
\hline
 \multirow[t]{4}{1.5cm}{Multirow Cell One} & Alpha &
 \multirow[b]{4}{1.5cm}{Multirow Cell Two} & Alpha \\
 & Beta  & & Beta \\
 & Gamma & & Gamma \\
 & Delta & & Delta \\
\hline
\end{tabular}
\end{demo}

\begin{demohigh}
\begin{tblr}{|l|l|l|l|}
\hline
 \multirow[t]{4}{1.5cm}{Multirow Cell One} & Alpha &
 \multirow[b]{4}{1.5cm}{Multirow Cell Two} & Alpha \\
 & Beta  & & Beta \\
 & Gamma & & Gamma \\
 & Delta & & Delta \\
\hline
\end{tblr}
\end{demohigh}

Note that you don't need to load \verb!multirow! package first,
since \verb!tabularrray! doesn't depend on it.
Furthermore, \verb!tabularray! will always typeset descent multirow cells.
First, it will set correct vertical \verb!c! alignment,
even though some rows have large height:

\begin{demo}
\begin{tabular}{|l|m{4em}|}
\hline
 \multirow[c]{4}{1.5cm}{Multirow} & Alpha  \\
 & Beta  \\
 & Gamma \\
 & Delta Delta Delta \\
\hline
\end{tabular}
\end{demo}

\begin{demohigh}
\begin{tblr}{|l|m{4em}|}
\hline
 \multirow[c]{4}{1.5cm}{Multirow} & Alpha  \\
 & Beta  \\
 & Gamma \\
 & Delta Delta Delta \\
\hline
\end{tblr}
\end{demohigh}

Second, it will enlarge row heights if the multirow cells have large height,
therefore it always avoids vertical overflow:

\begin{demo}
\begin{tabular}{|l|m{4em}|}
\hline
 \multirow[c]{2}{1cm}{Line \\ Line \\ Line \\ Line} & Alpha \\
\cline{2-2}
 & Beta \\
\hline
\end{tabular}
\end{demo}

\begin{demohigh}
\begin{tblr}{|l|m{4em}|}
\hline
 \multirow[c]{2}{1cm}{Line \\ Line \\ Line \\ Line} & Alpha \\
\cline{2}
 & Beta \\
\hline
\end{tblr}
\end{demohigh}

\section{Multi Rows and Columns}

It was a hard job to typeset cells with multiple rows and multiple columns. For example:

\begin{demo}
\begin{tabular}{|c|c|c|c|c|}
\hline
 \multirow{2}{*}{2 Rows}
     & \multicolumn{2}{c|}{2 Columns}
                 & \multicolumn{2}{c|}{\multirow{2}{*}{2 Rows 2 Columns}} \\
\cline{2-3}
     & 2-2 & 2-3 & \multicolumn{2}{c|}{} \\
\hline
 3-1 & 3-2 & 3-3 & 3-4 & 3-5 \\
\hline
\end{tabular}
\end{demo}

With \verb!tabularray! package, you can set spanned cells with \verb!\SetCell! command:
within the optional argument of \verb!\SetCell! command,
option \verb!r! is for rowspan number, and \verb!c! for colspan number;
within the mandatory argument of it, horizontal and vertical alignment options are accepted.
Therefore it's much simpler to typeset spanned cells:

\begin{demohigh}
\begin{tblr}{|c|c|c|c|c|}
\hline
 \SetCell[r=2]{c} 2 Rows
     & \SetCell[c=2]{c} 2 Columns
           &     & \SetCell[r=2,c=2]{c} 2 Rows 2 Columns & \\
\hline
     & 2-2 & 2-3 &     &     \\
\hline
 3-1 & 3-2 & 3-3 & 3-4 & 3-5 \\
\hline
\end{tblr}
\end{demohigh}

Using \verb!\multicolumn! command, the omitted cells \textcolor{red3}{must} be removed.
On the contrary,
using \verb!\multirow! command, the omitted cells \textcolor{red3}{must not} be removed.
\verb!\SetCell! command behaves the same as \verb!\multirow! command in this aspect.

With \verb!tblr! environment, any \verb!\hline! segments inside a spanned cell will be ignored,
therefore we're free to use \verb!\hline! in the above example.
Also, any omitted cell will definitely be ignored when typesetting,
no matter it's empty or not.
With this feature, we could put row and column numbers into the omitted cells,
which will help us to locate cells when the tables are rather complex:

\begin{demohigh}
\begin{tblr}{|ll|c|rr|}
\hline
 \SetCell[r=3,c=2]{h} r=3 c=2 & 1-2 & \SetCell[r=2,c=3]{r} r=2 c=3 & 1-4 & 1-5 \\ 
 2-1 & 2-2 & 2-3 & 2-4 & 2-5 \\
\hline
 3-1 & 3-2 & MIDDLE & \SetCell[r=3,c=2]{f} r=3 c=2 & 3-5 \\
\hline
 \SetCell[r=2,c=3]{l} r=2 c=3 & 4-2 & 4-3 & 4-4 & 4-5 \\
 5-1 & 5-2 & 5-3 & 5-4 & 5-5 \\
\hline
\end{tblr}
\end{demohigh}

\section{Column Types}

\verb!Tabularray! package supports all normal column types, as well as
the extendable \verb!X! column type,
which first occured in \verb!tabularx! package and was largely improved by \verb!tabu! package:

\begin{demohigh}
\begin{tblr}{|X[2,l]|X[3,l]|X[1,r]|X[r]|}
\hline
 Alpha & Beta & Gamma & Delta \\
\hline
\end{tblr}
\end{demohigh}

Also, \verb!X! columns with negative coefficients are possible:

\begin{demohigh}
\begin{tblr}{|X[2,l]|X[3,l]|X[-1,r]|X[r]|}
\hline
 Alpha & Beta & Gamma & Delta \\
\hline
\end{tblr}
\end{demohigh}

We need the width to typeset a table with \verb!X! columns.
If unset, the default is \verb!\linewidth!.
To change the width, we have to first put all column specifications into \verb!colspec={...}!:

\begin{demohigh}
\begin{tblr}{width=0.8\linewidth,colspec={|X[2,l]|X[3,l]|X[-1,r]|X[r]|}}
\hline
 Alpha & Beta & Gamma & Delta \\
\hline
\end{tblr}
\end{demohigh}

You can define new column types with \verb!\NewColumnType! command.
For example, in \verb!tabularray! package,
\verb!b! and \verb!X! columns are defined as special \verb!Q! columns:
\begin{codehigh}
\NewColumnType{b}[1]{Q[b,wd=#1]}
\NewColumnType{X}[1][]{Q[co=1,#1]}
\end{codehigh}

\section{Row Types}

Now that we have column types and \verb!colspec! option,
you may ask for row types and \verb!rowspec! option.
Yes, they are here:

\begin{demohigh}
\begin{tblr}{colspec={Q[l]Q[c]Q[r]},rowspec={|Q[t]|Q[m]|Q[b]|}}
 {Alpha \\ Alpha} & Beta               & Gamma \\
 Delta            & Epsilon            & {Zeta \\ Zeta}  \\
 Eta              & {Theta \\ Theta}   & Iota  \\
\end{tblr}
\end{demohigh}

Same as column types, \verb!Q! is the only primitive row type,
and other row types are defined as \verb!Q! types with different options.
It's better to specify horizontal alignment in \verb!colspec!,
and vertical alignment in \verb!rowspec!, respectively.

Inside \verb!rowspec!, \verb!|! is the hline type.
Therefore we need not to write \verb!\hline! commnad, which makes table code cleaner.

\section{Hlines and Vlines}

Hlines and vlines have been improved too. You can specify the widths and styles of them:

\begin{demohigh}
\begin{tblr}{|l|[dotted]|[2pt]c|r|[solid]|[dashed]|}
\hline
One   &  Two  & Three \\
\hline\hline[dotted]\hline
Four  & Five  &   Six \\
\hline[dashed]\hline[1pt]
Seven & Eight &  Nine \\
\hline
\end{tblr}
\end{demohigh}

\section{Colorful Tables}

To add colors to your tables, you need to load \verb!xcolor! package first.
\verb!Tabularray! package will also load \verb!ninecolors! package for proper color contrast.
First you can specify background option for \verb!Q! rows/columns inside \verb!rowspec!/\verb!colspec!:

\begin{demohigh}
\begin{tblr}{colspec={lcr},rowspec={|Q[cyan7]|Q[azure7]|Q[blue7]|}}
 Alpha   & Beta  & Gamma  \\
 Epsilon & Zeta  & Eta    \\
 Iota    & Kappa & Lambda \\
\end{tblr}
\end{demohigh}

\begin{demohigh}
\begin{tblr}{colspec={Q[l,brown7]Q[c,yellow7]Q[r,olive7]},rowspec={|Q|Q|Q|}}
 Alpha   & Beta  & Gamma  \\
 Epsilon & Zeta  & Eta    \\
 Iota    & Kappa & Lambda \\
\end{tblr}
\end{demohigh}

Also you can use \verb!\SetRow! or \verb!\SetColumn! command to specify row or column colors:

\begin{demohigh}
\begin{tblr}{colspec={lcr},rowspec={|Q|Q|Q|}}
 \SetRow{cyan7}  Alpha   & Beta  & Gamma  \\
 \SetRow{azure7} Epsilon & Zeta  & Eta    \\
 \SetRow{blue7}  Iota    & Kappa & Lambda \\
\end{tblr}
\end{demohigh}

\begin{demohigh}
\begin{tblr}{colspec={lcr},rowspec={|Q|Q|Q|}}
 \SetColumn{brown7}
 Alpha          & \SetColumn{yellow7}
                  Beta            & \SetColumn{olive7}
                                    Gamma  \\
 Epsilon        & Zeta            & Eta    \\
 Iota           & Kappa           & Lambda \\
\end{tblr}
\end{demohigh}

Hlines and vlines can also have colors:

\begin{demohigh}
\begin{tblr}{colspec={lcr},rowspec={|[2pt,green7]Q|[teal7]Q|[green7]Q|[3pt,teal7]}}
 Alpha   & Beta  & Gamma  \\
 Epsilon & Zeta  & Eta    \\
 Iota    & Kappa & Lambda \\
\end{tblr}
\end{demohigh}

\begin{demohigh}
\begin{tblr}{colspec={|[2pt,violet7]l|[2pt,magenta7]c|[2pt,purple7]r|[2pt,red7]}}
 Alpha   & Beta  & Gamma  \\
 Epsilon & Zeta  & Eta    \\
 Iota    & Kappa & Lambda \\
\end{tblr}
\end{demohigh}

\section{New Table Commands}

All commands which change the specifications of tables \textcolor{red3}{must} be defined with \verb!\NewTableCommand!.
The following example demonstrates how to define similar rules as in \verb!booktabs! package:

\begin{demohigh}
\NewTableCommand\toprule{\hline[0.08em]}
\NewTableCommand\midrule{\hline[0.05em]}
\NewTableCommand\bottomrule{\hline[0.08em]}
\begin{tblr}{llll}
\toprule
 Alpha   & Beta  & Gamma   & Delta \\
\midrule
 Epsilon & Zeta  & Eta     & Theta \\
 Iota    & Kappa & Lambda  & Mu    \\
 Nu      & Xi    & Omicron & Pi    \\
\bottomrule
\end{tblr}
\end{demohigh}

\chapter{New Interface}

With \verb!tabularray! package, you can separate style and content totally in tables.

\section{Hlines and Vlines}

Options \verb!hlines! and \verb!vlines! are for setting all hlines and vlines, respectively.
With empty value, all hlines/vlines will be solid.

\begin{demohigh}
\begin{tblr}{hlines,vlines}
 Alpha   & Beta  & Gamma   & Delta   \\
 Epsilon & Zeta  & Eta     & Theta   \\
 Iota    & Kappa & Lambda  & Mu      \\
 Nu      & Xi    & Omicron & Pi      \\
 Rho     & Sigma & Tau     & Upsilon \\
 Phi     & Chi   & Psi     & Omega   \\
\end{tblr}
\end{demohigh}

With values inside one pair of braces, all hlines/vlines will be styled.

\begin{demohigh}
\begin{tblr}{
 hlines = {1pt,solid},
 vlines = {red3,dashed},
}
 Alpha   & Beta  & Gamma   & Delta   \\
 Epsilon & Zeta  & Eta     & Theta   \\
 Iota    & Kappa & Lambda  & Mu      \\
 Nu      & Xi    & Omicron & Pi      \\
 Rho     & Sigma & Tau     & Upsilon \\
 Phi     & Chi   & Psi     & Omega   \\
\end{tblr}
\end{demohigh}

Another pair of braces before will select segments in all hlines/vlines.

\begin{demohigh}
\begin{tblr}{
 vlines = {1,3,5}{dashed},
 vlines = {2,4,6}{solid},
}
 Alpha   & Beta  & Gamma   & Delta   \\
 Epsilon & Zeta  & Eta     & Theta   \\
 Iota    & Kappa & Lambda  & Mu      \\
 Nu      & Xi    & Omicron & Pi      \\
 Rho     & Sigma & Tau     & Upsilon \\
 Phi     & Chi   & Psi     & Omega   \\
\end{tblr}
\end{demohigh}

The above example can be simplified with \verb!odd! and \verb!even! values.
(More child selectors can be defined with \verb!\NewChildSelector! command.
Advanced users could read the source code for this.)

\begin{demohigh}
\begin{tblr}{
 vlines = {odd}{dashed},
 vlines = {even}{solid},
}
 Alpha   & Beta  & Gamma   & Delta   \\
 Epsilon & Zeta  & Eta     & Theta   \\
 Iota    & Kappa & Lambda  & Mu      \\
 Nu      & Xi    & Omicron & Pi      \\
 Rho     & Sigma & Tau     & Upsilon \\
 Phi     & Chi   & Psi     & Omega   \\
\end{tblr}
\end{demohigh}

Another pair of braces before will draw more hlines/vlines (in which \verb!-! stands for all line segments).

\begin{demohigh}
\begin{tblr}{
 hlines = {1}{-}{dashed},
 hlines = {2}{-}{solid},
}
 Alpha   & Beta  & Gamma   & Delta   \\
 Epsilon & Zeta  & Eta     & Theta   \\
 Iota    & Kappa & Lambda  & Mu      \\
 Nu      & Xi    & Omicron & Pi      \\
 Rho     & Sigma & Tau     & Upsilon \\
 Phi     & Chi   & Psi     & Omega   \\
\end{tblr}
\end{demohigh}

Options \verb!hline{i}! and \verb!vline{j}! are for setting some hlines and vlines, respectively.

\begin{demohigh}
\begin{tblr}{
 hline{1,7} = {1pt,solid},
 hline{3-5} = {blue3,dashed},
 vline{1,5} = {3-4}{dotted},
}
 Alpha   & Beta  & Gamma   & Delta   \\
 Epsilon & Zeta  & Eta     & Theta   \\
 Iota    & Kappa & Lambda  & Mu      \\
 Nu      & Xi    & Omicron & Pi      \\
 Rho     & Sigma & Tau     & Upsilon \\
 Phi     & Chi   & Psi     & Omega   \\
\end{tblr}
\end{demohigh}

\section{Cells and Spancells}

Option \verb!cells! is for setting all cells.

\begin{demohigh}
\begin{tblr}{hlines,cells={c,blue7}}
 Alpha   & Beta  & Gamma   & Delta   \\
 Epsilon & Zeta  & Eta     & Theta   \\
 Iota    & Kappa & Lambda  & Mu      \\
 Nu      & Xi    & Omicron & Pi      \\
 Rho     & Sigma & Tau     & Upsilon \\
 Phi     & Chi   & Psi     & Omega   \\
\end{tblr}
\end{demohigh}

Option \verb!cell{i}{j}! is for setting some cells.

\begin{demohigh}
\begin{tblr}{
 hlines,
 cell{odd}{1,4} = {red7},
 cell{even}{1,4} = {purple7},
 cell{2}{2} = {r=4,c=2}{c,azure7},
}
 Alpha   & Beta  & Gamma   & Delta   \\
 Epsilon & Zeta  & Eta     & Theta   \\
 Iota    & Kappa & Lambda  & Mu      \\
 Nu      & Xi    & Omicron & Pi      \\
 Rho     & Sigma & Tau     & Upsilon \\
 Phi     & Chi   & Psi     & Omega   \\
\end{tblr}
\end{demohigh}

\section{Rows and Columns}

Options \verb!rows! and \verb!columns! are for setting all rows and columns, respectively.

\begin{demohigh}
\begin{tblr}{
 hlines,
 vlines,
 rows = {7mm},
 columns = {15mm,c}
}
 Alpha   & Beta  & Gamma   & Delta   \\
 Epsilon & Zeta  & Eta     & Theta   \\
 Iota    & Kappa & Lambda  & Mu      \\
 Nu      & Xi    & Omicron & Pi      \\
 Rho     & Sigma & Tau     & Upsilon \\
 Phi     & Chi   & Psi     & Omega   \\
\end{tblr}
\end{demohigh}

Options \verb!row{i}! and \verb!column{j}! are for setting some rows and columns, respectively.

\begin{demohigh}
\begin{tblr}{
 hlines = {1pt,white},
 row{odd} = {blue7},
 row{even} = {azure7},
 column{1} = {purple7,c},
}
 Alpha   & Beta  & Gamma   & Delta   \\
 Epsilon & Zeta  & Eta     & Theta   \\
 Iota    & Kappa & Lambda  & Mu      \\
 Nu      & Xi    & Omicron & Pi      \\
 Rho     & Sigma & Tau     & Upsilon \\
 Phi     & Chi   & Psi     & Omega   \\
\end{tblr}
\end{demohigh}

\section{Space in Tables}

Options \verb!rowsep! and \verb!colsep! are for setting padding for rows and columns, respectively.

\begin{demohigh}
\SetTblrDefault{rowsep=2pt,colsep=2pt}
\begin{tblr}{hlines,vlines}
 Alpha   & Beta  & Gamma  & Delta \\
 Epsilon & Zeta  & Eta    & Theta \\
 Iota    & Kappa & Lambda & Mu    \\
\end{tblr}
\end{demohigh}

Also \verb!abovesep!, \verb!belowsep!, \verb!leftsep!, \verb!rightsep! options are available:

\begin{demohigh}
\begin{tblr}{
 hlines,
 vlines,
 rows = {abovesep=1pt,belowsep=5pt},
 columns = {leftsep=1pt,rightsep=5pt},
}
 Alpha   & Beta  & Gamma  & Delta \\
 Epsilon & Zeta  & Eta    & Theta \\
 Iota    & Kappa & Lambda & Mu    \\
\end{tblr}
\end{demohigh}

And \verb!\\[dimen]! can be replaced by \verb!belowsep+! option:

\begin{demohigh}
\begin{tblr}{
 hlines, row{2} = {belowsep+=5pt},
}
 Alpha   & Beta  & Gamma  & Delta \\
 Epsilon & Zeta  & Eta    & Theta \\
 Iota    & Kappa & Lambda & Mu    \\
\end{tblr}
\end{demohigh}

Also \verb!\arraystretch! parameter can be replaced by \verb!stretch! option:

\begin{demohigh}
\begin{tblr}{hlines,stretch=1.5}
 Alpha   & Beta  & Gamma  & Delta \\
 Epsilon & Zeta  & Eta    & Theta \\
 Iota    & Kappa & Lambda & Mu    \\
\end{tblr}
\end{demohigh}

\section{Counters in Tables}

Counters \verb!rownum!, \verb!colnum!, \verb!rowcount!, \verb!colcount! can be used in cell text:

\begin{demohigh}
\begin{tblr}{hlines}
 Cell[\arabic{rownum}][\arabic{colnum}] & Cell[\arabic{rownum}][\arabic{colnum}] &
 Cell[\arabic{rownum}][\arabic{colnum}] & Cell[\arabic{rownum}][\arabic{colnum}] \\
 Cell[\arabic{rownum}][\arabic{colnum}] & Cell[\arabic{rownum}][\arabic{colnum}] &
 Cell[\arabic{rownum}][\arabic{colnum}] & Cell[\arabic{rownum}][\arabic{colnum}] \\
 Row=\arabic{rowcount},Col=\arabic{colcount} &
 Row=\arabic{rowcount},Col=\arabic{colcount} &
 Row=\arabic{rowcount},Col=\arabic{colcount} &
 Row=\arabic{rowcount},Col=\arabic{colcount} \\
 Cell[\arabic{rownum}][\arabic{colnum}] & Cell[\arabic{rownum}][\arabic{colnum}] &
 Cell[\arabic{rownum}][\arabic{colnum}] & Cell[\arabic{rownum}][\arabic{colnum}] \\
 Cell[\arabic{rownum}][\arabic{colnum}] & Cell[\arabic{rownum}][\arabic{colnum}] &
 Cell[\arabic{rownum}][\arabic{colnum}] & Cell[\arabic{rownum}][\arabic{colnum}] \\
\end{tblr}
\end{demohigh}

\section{Experimental Interface}

Everything described in this section is in \underline{\textcolor{red3}{\textbf{experimental}}} status.
Don’t use it in important documents, unless you have time
to update them for the newer versions of \verb!tabularray! package in the future.

By default \verb!tabularray! package will compute column widths from span widths.
By setting option \verb!hspan=minimal!, it will compute span widths from column widths.
The following two examples show this difference:

\begin{demohigh}
\begin{tblr}{|l|l|l|}
\hline
 111 111 & 222 222 & 333 333 \\
\hline
 \SetCell[c=2]{l} 12 Multi Columns Multi Columns 12 & & 333 \\
\hline
 \SetCell[c=3]{l} 13 Multi Columns Multi Columns Multi Columns 13 & & \\
\hline
 111 & \SetCell[c=2]{l} 23 Multi Columns Multi Columns 23 & \\
\hline
\end{tblr}
\end{demohigh}

\begin{demohigh}
\begin{tblr}{hspan=minimal,colspec={|l|l|l|}}
\hline
 111 111 & 222 222 & 333 333 \\
\hline
 \SetCell[c=2]{l} 12 Multi Columns Multi Columns 12 & & 333 \\
\hline
 \SetCell[c=3]{l} 13 Multi Columns Multi Columns Multi Columns 13 & & \\
\hline
 111 & \SetCell[c=2]{l} 23 Multi Columns Multi Columns 23 & \\
\hline
\end{tblr}
\end{demohigh}

\chapter{Source Code}

%\CodeHigh{lite}
%\dochighinput[language=latex/latex3]{tabularray.sty}

\end{document}
