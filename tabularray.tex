% -*- coding: utf-8 -*-
% !TEX program = lualatex
\documentclass[oneside]{book}
\usepackage[a4paper,margin=2.5cm]{geometry}

\usepackage{codehigh} % https://ctan.org/pkg/codehigh
\usepackage{tabularray}
\usepackage{array,multirow,amsmath}

\UseTblrLibrary{booktabs,diagbox}

\usepackage{hyperref}
\hypersetup{
  colorlinks=true,
  urlcolor=blue3,
  linkcolor=red3,
}

\usepackage{tcolorbox}
\tcbset{sharp corners, boxrule=0.5pt, colback=red9}

\usepackage{float}

\setcounter{tocdepth}{1}

\newcommand*{\K}[1]{\texttt{#1}}
\newcommand*{\V}[1]{\texttt{#1}}

\newcommand*{\experimental}{{\color{red3}\bfseries experimental}\space}

\renewcommand*{\thefootnote}{*}

\newcommand*{\myversion}{2021K}
\newcommand*{\mydate}{Version \myversion\ (\the\year-\mylpad\month-\mylpad\day)\\\myrepo}
\newcommand*{\myrepo}{\url{https://github.com/lvjr/tabularray}}
\newcommand*{\mylpad}[1]{\ifnum#1<10 0\the#1\else\the#1\fi}

\colorlet{highback}{\ifodd\month azure9\else blue9\fi}
\CodeHigh{language=latex/table,style/main=highback,style/code=highback}
\NewCodeHighEnv{code}{style/main=gray9,style/code=gray9}
\NewCodeHighEnv{demo}{style/main=gray9,style/code=gray9,demo}

%\CodeHigh{lite}

\begin{document}

\title{\sffamily\color{red3}Tabularray: Typeset Tabulars and Arrays with \LaTeX3}
\author{Jianrui Lyu (tolvjr@163.com)}
\date{\mydate}
\maketitle

\tableofcontents

\chapter{Overview of Features}

\section{Vertical Space}

After loading \verb!tabularray! package in the preamble,
we can use \verb!tblr! environments to typeset tabulars and arrays.
The name \verb!tblr! is short for \verb!tabularray! or \verb!top-bottom-left-right!.
The following is our first example:

\begin{demo}
\begin{tabular}{lccr}
\hline
 Alpha   & Beta  & Gamma  & Delta \\
\hline
 Epsilon & Zeta  & Eta    & Theta \\
\hline
 Iota    & Kappa & Lambda & Mu    \\
\hline
\end{tabular}
\end{demo}

\begin{demohigh}
\begin{tblr}{lccr}
\hline
 Alpha   & Beta  & Gamma  & Delta \\
\hline
 Epsilon & Zeta  & Eta    & Theta \\
\hline
 Iota    & Kappa & Lambda & Mu    \\
\hline
\end{tblr}
\end{demohigh}

You may notice that there is extra space above and below the table rows with \verb!tblr! envirenment.
This space makes the table look better.
If you don't like it, you could use \verb!\SetTblrInner! command:

\begin{demohigh}
\SetTblrInner{rowsep=0pt}
\begin{tblr}{lccr}
\hline
 Alpha   & Beta  & Gamma  & Delta \\
\hline
 Epsilon & Zeta  & Eta    & Theta \\
\hline
 Iota    & Kappa & Lambda & Mu    \\
\hline
\end{tblr}
\end{demohigh} 

But in many cases, this rowsep is useful:

\begin{demo}
$\begin{array}{rrr}
\hline
 \dfrac{2}{3} &  \dfrac{2}{3} &  \dfrac{1}{3} \\
 \dfrac{2}{3} & -\dfrac{1}{3} & -\dfrac{2}{3} \\
 \dfrac{1}{3} & -\dfrac{2}{3} &  \dfrac{2}{3} \\
\hline
\end{array}$
\end{demo}

\begin{demohigh}
$\begin{tblr}{rrr}
\hline
 \dfrac{2}{3} &  \dfrac{2}{3} &  \dfrac{1}{3} \\
 \dfrac{2}{3} & -\dfrac{1}{3} & -\dfrac{2}{3} \\
 \dfrac{1}{3} & -\dfrac{2}{3} &  \dfrac{2}{3} \\
\hline
\end{tblr}$
\end{demohigh}

Note that you can use \verb!tblr! in both text and math modes.

\section{Multiline Cells}

It's quite easy to write multiline cells without fixing the column width in \verb!tblr! environments:
just enclose the cell text with braces and use \verb!\\! to break lines:

\begin{demohigh}
\begin{tblr}{|l|c|r|}
\hline
 Left & {Center \\ Cent \\ C} & {Right \\ R} \\
\hline
 {L \\ Left} & {C \\ Cent \\ Center} & R \\
\hline
\end{tblr}
\end{demohigh} 

\section{Cell Alignment}

From time to time,
you may want to specify the horizontal and vertical alignment of cells at the same time.
\verb!Tabularray! package provides a \verb!Q! column for this
(In fact, \verb!Q! column is the only primitive column,
other columns are defined as \verb!Q! columns with some options):

\begin{demohigh}
\begin{tblr}{|Q[l,t]|Q[c,m]|Q[r,b]|}
\hline
 {Top Baseline \\ Left Left} & Middle Center & {Right Right \\ Bottom Baseline} \\
\hline
\end{tblr}
\end{demohigh} 

Note that you can use more meaningful \verb!t! instead of \verb!p! for top baseline alignment.
For some users who are familiar with word processors,
these \verb!t! and \verb!b! columns are counter-intuitive.
In \verb!tabularray! package, there are another two column types \verb!h! and \verb!f!,
which will align cell text at row head and row foot, respectively:

\begin{demohigh}
\begin{tblr}{Q[h,4em]Q[t,4em]Q[m,4em]Q[b,4em]Q[f,4em]}
\hline
 {row\\head} & {top\\line} & {middle} & {line\\bottom} & {row\\foot} \\
\hline
 {row\\head} & {top\\line} & {11\\22\\mid\\44\\55} & {line\\bottom} & {row\\foot} \\
\hline
\end{tblr}
\end{demohigh}

\section{Multirow Cells}

The above \verb!h! and \verb!f! columns are necessary for multirow cells.
In \verb!tabularray!, the \verb!t! and \verb!b! in the optional argument of \verb!\multirow! commands
will be treated as \verb!h! and \verb!f!, respectively:

\begin{demo}
\begin{tabular}{|l|l|l|l|}
\hline
 \multirow[t]{4}{1.5cm}{Multirow Cell One} & Alpha &
 \multirow[b]{4}{1.5cm}{Multirow Cell Two} & Alpha \\
 & Beta  & & Beta \\
 & Gamma & & Gamma \\
 & Delta & & Delta \\
\hline
\end{tabular}
\end{demo}

\begin{demohigh}
\begin{tblr}{|l|l|l|l|}
\hline
 \multirow[t]{4}{1.5cm}{Multirow Cell One} & Alpha &
 \multirow[b]{4}{1.5cm}{Multirow Cell Two} & Alpha \\
 & Beta  & & Beta \\
 & Gamma & & Gamma \\
 & Delta & & Delta \\
\hline
\end{tblr}
\end{demohigh}

Note that you don't need to load \verb!multirow! package first,
since \verb!tabularray! doesn't depend on it.
Furthermore, \verb!tabularray! will always typeset decent multirow cells.
First, it will set correct vertical \verb!c! alignment,
even though some rows have large height:

\begin{demo}
\begin{tabular}{|l|m{4em}|}
\hline
 \multirow[c]{4}{1.5cm}{Multirow} & Alpha  \\
 & Beta  \\
 & Gamma \\
 & Delta Delta Delta \\
\hline
\end{tabular}
\end{demo}

\begin{demohigh}
\begin{tblr}{|l|m{4em}|}
\hline
 \multirow[c]{4}{1.5cm}{Multirow} & Alpha  \\
 & Beta  \\
 & Gamma \\
 & Delta Delta Delta \\
\hline
\end{tblr}
\end{demohigh}

Second, it will enlarge row heights if the multirow cells have large height,
therefore it always avoids vertical overflow:

\begin{demo}
\begin{tabular}{|l|m{4em}|}
\hline
 \multirow[c]{2}{1cm}{Line \\ Line \\ Line \\ Line} & Alpha \\
\cline{2-2}
 & Beta \\
\hline
\end{tabular}
\end{demo}

\begin{demohigh}
\begin{tblr}{|l|m{4em}|}
\hline
 \multirow[c]{2}{1cm}{Line \\ Line \\ Line \\ Line} & Alpha \\
\cline{2}
 & Beta \\
\hline
\end{tblr}
\end{demohigh}

\section{Multi Rows and Columns}

It was a hard job to typeset cells with multiple rows and multiple columns. For example:

\begin{demo}
\begin{tabular}{|c|c|c|c|c|}
\hline
 \multirow{2}{*}{2 Rows}
     & \multicolumn{2}{c|}{2 Columns}
                 & \multicolumn{2}{c|}{\multirow{2}{*}{2 Rows 2 Columns}} \\
\cline{2-3}
     & 2-2 & 2-3 & \multicolumn{2}{c|}{} \\
\hline
 3-1 & 3-2 & 3-3 & 3-4 & 3-5 \\
\hline
\end{tabular}
\end{demo}

With \verb!tabularray! package, you can set spanned cells with \verb!\SetCell! command:
within the optional argument of \verb!\SetCell! command,
option \verb!r! is for rowspan number, and \verb!c! for colspan number;
within the mandatory argument of it, horizontal and vertical alignment options are accepted.
Therefore it's much simpler to typeset spanned cells:

\begin{demohigh}
\begin{tblr}{|c|c|c|c|c|}
\hline
 \SetCell[r=2]{c} 2 Rows
     & \SetCell[c=2]{c} 2 Columns
           &     & \SetCell[r=2,c=2]{c} 2 Rows 2 Columns & \\
\hline
     & 2-2 & 2-3 &     &     \\
\hline
 3-1 & 3-2 & 3-3 & 3-4 & 3-5 \\
\hline
\end{tblr}
\end{demohigh}

Using \verb!\multicolumn! command, the omitted cells \textcolor{red3}{must} be removed.
On the contrary,
using \verb!\multirow! command, the omitted cells \textcolor{red3}{must not} be removed.
\verb!\SetCell! command behaves the same as \verb!\multirow! command in this aspect.

With \verb!tblr! environment, any \verb!\hline! segments inside a spanned cell will be ignored,
therefore we're free to use \verb!\hline! in the above example.
Also, any omitted cell will definitely be ignored when typesetting,
no matter it's empty or not.
With this feature, we could put row and column numbers into the omitted cells,
which will help us to locate cells when the tables are rather complex:

\begin{demohigh}
\begin{tblr}{|ll|c|rr|}
\hline
 \SetCell[r=3,c=2]{h} r=3 c=2 & 1-2 & \SetCell[r=2,c=3]{r} r=2 c=3 & 1-4 & 1-5 \\ 
 2-1 & 2-2 & 2-3 & 2-4 & 2-5 \\
\hline
 3-1 & 3-2 & MIDDLE & \SetCell[r=3,c=2]{f} r=3 c=2 & 3-5 \\
\hline
 \SetCell[r=2,c=3]{l} r=2 c=3 & 4-2 & 4-3 & 4-4 & 4-5 \\
 5-1 & 5-2 & 5-3 & 5-4 & 5-5 \\
\hline
\end{tblr}
\end{demohigh}

\section{Column Types}

\verb!Tabularray! package supports all normal column types, as well as
the extendable \verb!X! column type,
which first occurred in \verb!tabularx! package and was largely improved by \verb!tabu! package:

\begin{demohigh}
\begin{tblr}{|X[2,l]|X[3,l]|X[1,r]|X[r]|}
\hline
 Alpha & Beta & Gamma & Delta \\
\hline
\end{tblr}
\end{demohigh}

Also, \verb!X! columns with negative coefficients are possible:

\begin{demohigh}
\begin{tblr}{|X[2,l]|X[3,l]|X[-1,r]|X[r]|}
\hline
 Alpha & Beta & Gamma & Delta \\
\hline
\end{tblr}
\end{demohigh}

We need the width to typeset a table with \verb!X! columns.
If unset, the default is \verb!\linewidth!.
To change the width, we have to first put all column specifications into \verb!colspec={...}!:

\begin{demohigh}
\begin{tblr}{width=0.8\linewidth,colspec={|X[2,l]|X[3,l]|X[-1,r]|X[r]|}}
\hline
 Alpha & Beta & Gamma & Delta \\
\hline
\end{tblr}
\end{demohigh}

You can define new column types with \verb!\NewColumnType! command.
For example, in \verb!tabularray! package,
\verb!b! and \verb!X! columns are defined as special \verb!Q! columns:
\begin{codehigh}
\NewColumnType{b}[1]{Q[b,wd=#1]}
\NewColumnType{X}[1][]{Q[co=1,#1]}
\end{codehigh}

\section{Row Types}

Now that we have column types and \verb!colspec! option,
you may ask for row types and \verb!rowspec! option.
Yes, they are here:

\begin{demohigh}
\begin{tblr}{colspec={Q[l]Q[c]Q[r]},rowspec={|Q[t]|Q[m]|Q[b]|}}
 {Alpha \\ Alpha} & Beta               & Gamma \\
 Delta            & Epsilon            & {Zeta \\ Zeta}  \\
 Eta              & {Theta \\ Theta}   & Iota  \\
\end{tblr}
\end{demohigh}

Same as column types, \verb!Q! is the only primitive row type,
and other row types are defined as \verb!Q! types with different options.
It's better to specify horizontal alignment in \verb!colspec!,
and vertical alignment in \verb!rowspec!, respectively.

Inside \verb!rowspec!, \verb!|! is the hline type.
Therefore we need not to write \verb!\hline! commnad, which makes table code cleaner.

\section{Hlines and Vlines}

Hlines and vlines have been improved too. You can specify the widths and styles of them:

\begin{demohigh}
\begin{tblr}{|l|[dotted]|[2pt]c|r|[solid]|[dashed]|}
\hline
One   &  Two  & Three \\
\hline\hline[dotted]\hline
Four  & Five  &   Six \\
\hline[dashed]\hline[1pt]
Seven & Eight &  Nine \\
\hline
\end{tblr}
\end{demohigh}

\section{Colorful Tables}

To add colors to your tables, you need to load \verb!xcolor! package first.
\verb!Tabularray! package will also load \verb!ninecolors! package for proper color contrast.
First you can specify background option for \verb!Q! rows/columns inside \verb!rowspec!/\verb!colspec!:

\begin{demohigh}
\begin{tblr}{colspec={lcr},rowspec={|Q[cyan7]|Q[azure7]|Q[blue7]|}}
 Alpha   & Beta  & Gamma  \\
 Epsilon & Zeta  & Eta    \\
 Iota    & Kappa & Lambda \\
\end{tblr}
\end{demohigh}

\begin{demohigh}
\begin{tblr}{colspec={Q[l,brown7]Q[c,yellow7]Q[r,olive7]},rowspec={|Q|Q|Q|}}
 Alpha   & Beta  & Gamma  \\
 Epsilon & Zeta  & Eta    \\
 Iota    & Kappa & Lambda \\
\end{tblr}
\end{demohigh}

Also you can use \verb!\SetRow! or \verb!\SetColumn! command to specify row or column colors:

\begin{demohigh}
\begin{tblr}{colspec={lcr},rowspec={|Q|Q|Q|}}
 \SetRow{cyan7}  Alpha   & Beta  & Gamma  \\
 \SetRow{azure7} Epsilon & Zeta  & Eta    \\
 \SetRow{blue7}  Iota    & Kappa & Lambda \\
\end{tblr}
\end{demohigh}

\begin{demohigh}
\begin{tblr}{colspec={lcr},rowspec={|Q|Q|Q|}}
 \SetColumn{brown7}
 Alpha          & \SetColumn{yellow7}
                  Beta            & \SetColumn{olive7}
                                    Gamma  \\
 Epsilon        & Zeta            & Eta    \\
 Iota           & Kappa           & Lambda \\
\end{tblr}
\end{demohigh}

Hlines and vlines can also have colors:

\begin{demohigh}
\begin{tblr}{colspec={lcr},rowspec={|[2pt,green7]Q|[teal7]Q|[green7]Q|[3pt,teal7]}}
 Alpha   & Beta  & Gamma  \\
 Epsilon & Zeta  & Eta    \\
 Iota    & Kappa & Lambda \\
\end{tblr}
\end{demohigh}

\begin{demohigh}
\begin{tblr}{colspec={|[2pt,violet5]l|[2pt,magenta5]c|[2pt,purple5]r|[2pt,red5]}}
 Alpha   & Beta  & Gamma  \\
 Epsilon & Zeta  & Eta    \\
 Iota    & Kappa & Lambda \\
\end{tblr}
\end{demohigh}

\section{New Table Commands}

All commands which change the specifications of tables \textcolor{red3}{must} be defined with \verb!\NewTableCommand!.
The following example demonstrates how to define similar rules as in \verb!booktabs! package:

\begin{codehigh}
\NewTableCommand\toprule{\hline[0.08em]}
\NewTableCommand\midrule{\hline[0.05em]}
\NewTableCommand\bottomrule{\hline[0.08em]}
\begin{tblr}{llll}
\toprule
 Alpha   & Beta  & Gamma   & Delta \\
\midrule
 Epsilon & Zeta  & Eta     & Theta \\
 Iota    & Kappa & Lambda  & Mu    \\
 Nu      & Xi    & Omicron & Pi    \\
\bottomrule
\end{tblr}
\end{codehigh}

\chapter{The Old Interfaces}

With tabularray package, you can still use improved table commands to change the styles of tables.
Same as \verb!tabular! and \verb!array! environments,
all table commands \textcolor{red3}{must} be put at the beginning ot the cell text.
Also, new table commands \textcolor{red3}{must} be defined with \verb!\NewTableCommand!.

\section{Hline Commands}

The \verb!\hline! command has an optional argument which accepts key-value options.
The available keys are described in Table \ref{key:hvline}.

\begin{demohigh}
\begin{tblr}{llll}
\hline
 Alpha   & Beta  & Gamma  & Delta \\
\hline[dashed]
 Epsilon & Zeta  & Eta    & Theta \\
\hline[dotted]
 Iota    & Kappa & Lambda & Mu    \\
\hline[2pt,blue5]
\end{tblr}
\end{demohigh}

The \verb!\cline! command also has an optional argument which is the same as \verb!\hline!.

\begin{demohigh}
\begin{tblr}{llll}
\cline{1-4}
 Alpha   & Beta  & Gamma  & Delta \\
\cline[dashed]{1,3}
 Epsilon & Zeta  & Eta    & Theta \\
\cline[dashed]{2,4}
 Iota    & Kappa & Lambda & Mu    \\
\cline[2pt,blue5]{-}
\end{tblr}
\end{demohigh}

You can use child selectors in the mandatory argument of \verb!\cline!.

\begin{demohigh}
\begin{tblr}{llll}
\cline{1-4}
 Alpha   & Beta  & Gamma  & Delta \\
\cline[dashed]{odd}
 Epsilon & Zeta  & Eta    & Theta \\
\cline[dashed]{even}
 Iota    & Kappa & Lambda & Mu    \\
\cline[2pt,blue5]{-}
\end{tblr}
\end{demohigh}

\section{Cell Commands}

The \verb!\SetCell! command has a mandatory argument for setting the styles of current cell.
The available keys are described in Table \ref{key:cell}.

\begin{demohigh}
\begin{tblr}{llll}
\hline[1pt]
 Alpha   & \SetCell{bg=teal2,fg=white} Beta & Gamma & Delta \\
\hline
 Epsilon & Zeta & \SetCell{r,font=\scshape} Eta & Theta \\
\hline
 Iota    & Kappa & Lambda & Mu    \\
\hline[1pt]
\end{tblr}
\end{demohigh}

The \verb!\SetCell! command also has an optional argument for setting the multispan of current cell.
The available keys are described in Table \ref{key:cellspan}.

\begin{demohigh}
\begin{tblr}{|X|X|X|X|X|X|}
\hline
 Alpha & Beta & Gamma & Delta & Epsilon & Zeta \\
\hline
 \SetCell[c=2]{c} Eta & 2-2
              & \SetCell[c=2]{c} Iota & 2-4
                              & \SetCell[c=2]{c} Lambda  & 2-6 \\
\hline
 \SetCell[c=3]{l} Nu & 3-2 & 3-3
                      & \SetCell[c=3]{l} Pi & 3-5 & 3-6   \\
\hline
 \SetCell[c=6]{r} Tau & 4-2 & 4-3 & 4-4 & 4-5 & 4-6 \\
\hline
\end{tblr}
\end{demohigh}

\section{Row Commands}

The \verb!\SetRow! command has a mandatory argument for setting the styles of current row.
The available keys are described in Table \ref{key:row}.

\begin{demohigh}
\begin{tblr}{llll}
\hline[1pt]
 \SetRow{azure8} Alpha & Beta & Gamma & Delta \\
\hline
 \SetRow{blue8,c} Epsilon & Zeta & Eta & Theta \\
\hline
 \SetRow{violet8} Iota & Kappa & Lambda & Mu \\
\hline[1pt]
\end{tblr}
\end{demohigh}

\section{Column Types}

The \verb!tabularray! package has only one type of primitive column: the \verb!Q! column.
Other types of columns are defined as \verb!Q! columns with some keys.

\begin{codehigh}
\NewColumnType{l}{Q[l]}
\NewColumnType{c}{Q[c]}
\NewColumnType{r}{Q[r]}
\NewColumnType{t}[1]{Q[t,wd=#1]}
\NewColumnType{m}[1]{Q[m,wd=#1]}
\NewColumnType{b}[1]{Q[b,wd=#1]}
\NewColumnType{h}[1]{Q[h,wd=#1]}
\NewColumnType{f}[1]{Q[f,wd=#1]}
\NewColumnType{X}[1][]{Q[co=1,#1]}
\end{codehigh}

\begin{demohigh}
\begin{tblr}{|t{15mm}|m{15mm}|b{20mm}|}
 Alpha   & Beta  & {Gamma\\Gamma} \\
 Epsilon & Zeta  & {Eta\\Eta} \\
 Iota    & Kappa & {Lambda\\Lambda} \\
\end{tblr}
\end{demohigh}

Any new column type must be defined with \verb!\NewColumnType! command.
It can have an optional argument when it's defined.

\section{Row Types}

The \verb!tabularray! package has only one type of primitive row: the \verb!Q! row.
Other types of rows are defined as \verb!Q! rows with some keys.

\begin{codehigh}
\NewRowType{l}{Q[l]}
\NewRowType{c}{Q[c]}
\NewRowType{r}{Q[r]}
\NewRowType{t}[1]{Q[t,ht=#1]}
\NewRowType{m}[1]{Q[m,ht=#1]}
\NewRowType{b}[1]{Q[b,ht=#1]}
\NewRowType{h}[1]{Q[h,ht=#1]}
\NewRowType{f}[1]{Q[f,ht=#1]}
\end{codehigh}

\begin{demohigh}
\begin{tblr}{rowspec={|t{12mm}|m{10mm}|b{10mm}|}}
 Alpha   & Beta  & {Gamma\\Gamma} \\
 Epsilon & Zeta  & {Eta\\Eta} \\
 Iota    & Kappa & {Lambda\\Lambda} \\
\end{tblr}
\end{demohigh}

Any new row type must be defined with \verb!\NewRowType! command.
It can have an optional argument when it's defined.

\chapter{The New Interfaces}

With \verb!tabularray! package, you can separate style and content totally in tables.

\section{Hlines and Vlines}

All available keys for hlines and vlines are described in Table \ref{key:hvline}.

\begin{longtblr}[
  caption = {Keys for Hlines and Vlines},
  label = {key:hvline},
  remark{Note} = {In most cases, you can omit the underlined key names and write only their values.}
]{
  hlines = {white},
  row{odd}={brown8},row{even}={gray8},row{1}={fg=white,bg=purple2,font=\bfseries\sffamily},
  column{2} = {co=1},
}
  Key & Description and Values & Initial Value \\
  \underline{\K{dash}}
    & dash style: \V{solid}, \V{dashed} or \V{dotted} & \V{solid} \\
  \underline{\K{wd}} & rule width dimension & None \\
  \underline{\K{fg}} & rule color name & None \\
\end{longtblr}

Options \verb!hlines! and \verb!vlines! are for setting all hlines and vlines, respectively.
With empty value, all hlines/vlines will be solid.

\begin{demohigh}
\begin{tblr}{hlines,vlines}
 Alpha   & Beta  & Gamma   & Delta   \\
 Epsilon & Zeta  & Eta     & Theta   \\
 Iota    & Kappa & Lambda  & Mu      \\
 Nu      & Xi    & Omicron & Pi      \\
 Rho     & Sigma & Tau     & Upsilon \\
 Phi     & Chi   & Psi     & Omega   \\
\end{tblr}
\end{demohigh}

With values inside one pair of braces, all hlines/vlines will be styled.

\begin{demohigh}
\begin{tblr}{
 hlines = {1pt,solid},
 vlines = {red3,dashed},
}
 Alpha   & Beta  & Gamma   & Delta   \\
 Epsilon & Zeta  & Eta     & Theta   \\
 Iota    & Kappa & Lambda  & Mu      \\
 Nu      & Xi    & Omicron & Pi      \\
 Rho     & Sigma & Tau     & Upsilon \\
 Phi     & Chi   & Psi     & Omega   \\
\end{tblr}
\end{demohigh}

Another pair of braces before will select segments in all hlines/vlines.

\begin{demohigh}
\begin{tblr}{
 vlines = {1,3,5}{dashed},
 vlines = {2,4,6}{solid},
}
 Alpha   & Beta  & Gamma   & Delta   \\
 Epsilon & Zeta  & Eta     & Theta   \\
 Iota    & Kappa & Lambda  & Mu      \\
 Nu      & Xi    & Omicron & Pi      \\
 Rho     & Sigma & Tau     & Upsilon \\
 Phi     & Chi   & Psi     & Omega   \\
\end{tblr}
\end{demohigh}

The above example can be simplified with \verb!odd! and \verb!even! values.
(More child selectors can be defined with \verb!\NewChildSelector! command.
Advanced users could read the source code for this.)

\begin{demohigh}
\begin{tblr}{
 vlines = {odd}{dashed},
 vlines = {even}{solid},
}
 Alpha   & Beta  & Gamma   & Delta   \\
 Epsilon & Zeta  & Eta     & Theta   \\
 Iota    & Kappa & Lambda  & Mu      \\
 Nu      & Xi    & Omicron & Pi      \\
 Rho     & Sigma & Tau     & Upsilon \\
 Phi     & Chi   & Psi     & Omega   \\
\end{tblr}
\end{demohigh}

Another pair of braces before will draw more hlines/vlines (in which \verb!-! stands for all line segments).

\begin{demohigh}
\begin{tblr}{
 hlines = {1}{-}{dashed},
 hlines = {2}{-}{solid},
}
 Alpha   & Beta  & Gamma   & Delta   \\
 Epsilon & Zeta  & Eta     & Theta   \\
 Iota    & Kappa & Lambda  & Mu      \\
 Nu      & Xi    & Omicron & Pi      \\
 Rho     & Sigma & Tau     & Upsilon \\
 Phi     & Chi   & Psi     & Omega   \\
\end{tblr}
\end{demohigh}

Options \verb!hline{i}! and \verb!vline{j}! are for setting some hlines and vlines, respectively.

\begin{demohigh}
\begin{tblr}{
 hline{1,7} = {1pt,solid},
 hline{3-5} = {blue3,dashed},
 vline{1,5} = {3-4}{dotted},
}
 Alpha   & Beta  & Gamma   & Delta   \\
 Epsilon & Zeta  & Eta     & Theta   \\
 Iota    & Kappa & Lambda  & Mu      \\
 Nu      & Xi    & Omicron & Pi      \\
 Rho     & Sigma & Tau     & Upsilon \\
 Phi     & Chi   & Psi     & Omega   \\
\end{tblr}
\end{demohigh}

\newpage %TODO: fix pagebreak bug

\section{Cells and Spancells}

All available keys for cells are described in Table \ref{key:cell} and Table \ref{key:cellspan}.
\nopagebreak
\begin{longtblr}[
  caption = {Keys for the Content of Cells},
  label = {key:cell},
  remark{Note} = {In most cases, you can omit the underlined key names and write only their values.}
]{
  hlines = {white},
  row{odd}={brown8},row{even}={gray8},row{1}={fg=white,bg=purple2,font=\bfseries\sffamily},
  column{2} = {co=1},
}
  Key & Description and Values & Initial Value \\
  \underline{\K{halign}}
    & horizontal alignment: \V{l} (left), \V{c} (center), or \V{r} (right)
    & \V{l} \\
  \underline{\K{valign}}
    & vertical alignment: \V{t} (top), \V{m} (middle), \V{b} (bottom),
      \V{h} (head) or \V{f} (foot)
    & \V{t} \\
  \underline{\K{wd}} & width dimension & None \\
  \underline{\K{bg}} & background color name & None \\
  \K{fg} & foreground color name & None \\
  \K{font} & font commands & None \\
  \K{preto} & prepend text to the cell & None \\
  \K{appto} & append text to the cell & None \\
\end{longtblr}
\vspace{-2em}
\begin{longtblr}[
  caption = {Keys for Multispan of Cells},
  label = {key:cellspan},
]{
  hlines = {white},
  row{odd}={brown8},row{even}={gray8},row{1}={fg=white,bg=purple2,font=\bfseries\sffamily},
  column{2} = {co=1},
}
  Key & Description and Values & Initial Value \\
  \K{r} & number of rows the cell spans    & 1 \\
  \K{c} & number of columns the cell spans & 1 \\
\end{longtblr}

Option \verb!cells! is for setting all cells.

\begin{demohigh}
\begin{tblr}{hlines={white},cells={c,blue7}}
 Alpha   & Beta  & Gamma   & Delta   \\
 Epsilon & Zeta  & Eta     & Theta   \\
 Iota    & Kappa & Lambda  & Mu      \\
 Nu      & Xi    & Omicron & Pi      \\
 Rho     & Sigma & Tau     & Upsilon \\
 Phi     & Chi   & Psi     & Omega   \\
\end{tblr}
\end{demohigh}

Option \verb!cell{i}{j}! is for setting some cells.

\begin{demohigh}
\begin{tblr}{
 hlines = {white},
 vlines = {white},
 cell{1,6}{odd} = {teal7},
 cell{1,6}{even} = {green7},
 cell{2,4}{1,4} = {red7},
 cell{3,5}{1,4} = {purple7},
 cell{2}{2} = {r=4,c=2}{c,azure7},
}
 Alpha   & Beta  & Gamma   & Delta   \\
 Epsilon & Zeta  & Eta     & Theta   \\
 Iota    & Kappa & Lambda  & Mu      \\
 Nu      & Xi    & Omicron & Pi      \\
 Rho     & Sigma & Tau     & Upsilon \\
 Phi     & Chi   & Psi     & Omega   \\
\end{tblr}
\end{demohigh}

\section{Rows and Columns}

All available keys for rows and columns are described in Table \ref{key:row} and Table \ref{key:column}.

\begin{longtblr}[
  caption = {Keys for Rows},
  label = {key:row},
  remark{Note} = {In most cases, you can omit the underlined key names and write only their values.}
]{
  rowhead = 1,
  hlines = {white},
  row{odd}={brown8},row{even}={gray8},row{1}={fg=white,bg=purple2,font=\bfseries\sffamily},
  column{2} = {co=1},
}
  Key & Description and Values & Initial Value \\
  \underline{\K{halign}}
    & horizontal alignment: \V{l} (left), \V{c} (center), or \V{r} (right)
    & \V{l} \\
  \underline{\K{valign}}
    & vertical alignment: \V{t} (top), \V{m} (middle), \V{b} (bottom),
      \V{h} (head) or \V{f} (foot)
    & \V{t} \\
  \underline{\K{ht}} & height dimension & None \\
  \underline{\K{bg}} & background color name & None \\
  \K{fg} & foreground color name & None \\
  \K{font} & font commands & None \\
  \K{abovesep} & set vertical space above the row & \V{2pt} \\
  \K{abovesep+} & increase vertical space above the row & None \\
  \K{belowsep} & set vertical space below the row & \V{2pt} \\
  \K{belowsep+} & increase vertical space below the row & None \\
  \K{rowsep} & set vertical space above and below the row & \V{2pt} \\
  \K{rowsep+} & increase vertical space above and below the row & None \\
\end{longtblr}
\vspace{-2em}
\begin{longtblr}[
  caption = {Keys for Columns},
  label = {key:column},
  remark{Note} = {In most cases, you can omit the underlined key names and write only their values.}
]{
  rowhead = 1,
  hlines = {white},
  row{odd}={brown8},row{even}={gray8},row{1}={fg=white,bg=purple2,font=\bfseries\sffamily},
  column{2} = {co=1},
}
  Key & Description and Values & Initial Value \\
  \underline{\K{halign}}
    & horizontal alignment: \V{l} (left), \V{c} (center), or \V{r} (right)
    & \V{l} \\
  \underline{\K{valign}}
    & vertical alignment: \V{t} (top), \V{m} (middle), \V{b} (bottom),
      \V{h} (head) or \V{f} (foot)
    & \V{t} \\
  \underline{\K{wd}} & width dimension & None \\
  \underline{\K{co}} & coefficient for the extendable column (\V{X} column) & None \\
  \underline{\K{bg}} & background color name & None \\
  \K{fg} & foreground color name & None \\
  \K{font} & font commands & None \\
  \K{leftsep} & set horizontal space to the left of the column & \V{6pt} \\
  \K{leftsep+} & increase horizontal space to the left of the column & None \\
  \K{rightsep} & set horizontal space to the right of the column & \V{6pt} \\
  \K{rightsep+} & increase horizontal space to the right of the column & None \\
  \K{colsep} & set horizontal space to both sides of the column & \V{6pt} \\
  \K{colsep+} & increase horizontal space to both sides of the column & None \\
\end{longtblr}

Options \verb!rows! and \verb!columns! are for setting all rows and columns, respectively.
\nopagebreak
\begin{demohigh}
\begin{tblr}{
 hlines,
 vlines,
 rows = {7mm},
 columns = {15mm,c},
}
 Alpha   & Beta  & Gamma   & Delta \\
 Epsilon & Zeta  & Eta     & Theta \\
 Iota    & Kappa & Lambda  & Mu    \\
\end{tblr}
\end{demohigh}

Options \verb!row{i}! and \verb!column{j}! are for setting some rows and columns, respectively.

\begin{demohigh}
\begin{tblr}{
 hlines = {1pt,white},
 row{odd} = {blue7},
 row{even} = {azure7},
 column{1} = {purple7,c},
}
 Alpha   & Beta  & Gamma   & Delta   \\
 Epsilon & Zeta  & Eta     & Theta   \\
 Iota    & Kappa & Lambda  & Mu      \\
 Nu      & Xi    & Omicron & Pi      \\
 Rho     & Sigma & Tau     & Upsilon \\
 Phi     & Chi   & Psi     & Omega   \\
\end{tblr}
\end{demohigh}

The following example demonstrates the usages of \verb!bg!, \verb!fg! and \verb!font! keys.

\begin{demohigh}
\begin{tblr}{
 row{odd} = {bg=azure8},
 row{1}   = {bg=azure3, fg=white, font=\sffamily},
}
 Alpha & Beta    & Gamma \\
 Delta & Epsilon & Zeta  \\
 Eta   & Theta   & Iota  \\
 Kappa & Lambda  & Mu    \\
 Nu Xi Omikron & Pi Rho Sigma & Tau Upsilon Phi \\
\end{tblr}
\end{demohigh}

The following example demonstrates the usages of
\verb!abovesep!, \verb!belowsep!, \verb!leftsep!, \verb!rightsep! keys.
\nopagebreak
\begin{demohigh}
\begin{tblr}{
 hlines,
 vlines,
 rows = {abovesep=1pt,belowsep=5pt},
 columns = {leftsep=1pt,rightsep=5pt},
}
 Alpha   & Beta  & Gamma  & Delta \\
 Epsilon & Zeta  & Eta    & Theta \\
 Iota    & Kappa & Lambda & Mu    \\
\end{tblr}
\end{demohigh}

The following example shows that we can replace \verb!\\[dimen]! with \verb!belowsep+! key.

\begin{demohigh}
\begin{tblr}{
 hlines, row{2} = {belowsep+=5pt},
}
 Alpha   & Beta  & Gamma  & Delta \\
 Epsilon & Zeta  & Eta    & Theta \\
 Iota    & Kappa & Lambda & Mu    \\
\end{tblr}
\end{demohigh}

\section{The Whole Table}

All available keys for the whole table are described in Table \ref{key:table}.

\begin{longtblr}[
  caption = {Keys for the Whole Table},
  label = {key:table},
]{
  rowhead = 1,
  hlines = {white},
  row{odd}={brown8},row{even}={gray8},row{1}={fg=white,bg=purple2,font=\bfseries\sffamily},
  column{2} = {co=1},
}
  Key & Description and Values & Initial Value \\
  \K{rowspec} & set row specifications with row type specifiers & None \\
  \K{colspec} & set column specifications with column type specifiers & None \\
  \K{width} & width of the table with extendable columns & None \\
  \K{rulesep} & space between two hlines or vlines & \V{2pt} \\
  \K{stretch} & stretch ratio for struts added to cell text & \V{1} \\
  \K{abovesep} & set vertical space above every row & \V{2pt} \\
  \K{belowsep} & set vertical space below every row & \V{2pt} \\
  \K{rowsep} & set vertical space above and below every row & \V{2pt} \\
  \K{leftsep} & set horizontal space to the left of every column & \V{6pt} \\
  \K{rightsep} & set horizontal space to the right of every column & \V{6pt} \\
  \K{colsep} & set horizontal space to both sides of every column & \V{6pt} \\
  \K{hspan} & horizontal span algorithm: \V{default}, \V{even}, or \V{minimal} & \V{default} \\
  \K{vspan} & vertical span algorithm: \V{default} or \V{even} & \V{default} \\
\end{longtblr}

The following example demonstrates the usage of \verb!width! key.
\nopagebreak
\begin{demohigh}
\begin{tblr}{width=0.8\textwidth, colspec={|l|X[2]|X[3]|X[-1]|}}
 Alpha   & Beta  & Gamma  & Delta \\
 Epsilon & Zeta  & Eta    & Theta \\
 Iota    & Kappa & Lambda & Mu    \\
\end{tblr}
\end{demohigh}

The following example shows that we can replace \verb!\doublerulesep! parameter with \verb!rulesep! key.
\nopagebreak
\begin{demohigh}
\begin{tblr}{
 colspec={||llll||},rowspec={|QQQ|},rulesep=4pt,
}
 Alpha   & Beta  & Gamma  & Delta \\
 Epsilon & Zeta  & Eta    & Theta \\
 Iota    & Kappa & Lambda & Mu    \\
\end{tblr}
\end{demohigh}

The following example shows that we can replace \verb!\arraystretch! parameter with \verb!stretch! key.

\begin{demohigh}
\begin{tblr}{hlines,stretch=1.5}
 Alpha   & Beta  & Gamma  & Delta \\
 Epsilon & Zeta  & Eta    & Theta \\
 Iota    & Kappa & Lambda & Mu    \\
\end{tblr}
\end{demohigh}

The following example uses \verb!rowsep! and \verb!colsep! keys to set padding for all rows and columns.
\nopagebreak
\begin{demohigh}
\SetTblrInner{rowsep=2pt,colsep=2pt}
\begin{tblr}{hlines,vlines}
 Alpha   & Beta  & Gamma  & Delta \\
 Epsilon & Zeta  & Eta    & Theta \\
 Iota    & Kappa & Lambda & Mu    \\
\end{tblr}
\end{demohigh}

With \verb!hspan=default! or \verb!hspan=even!,
\verb!tabularray! package will compute column widths from span widths.
But with \verb!hspan=minimal!, it will compute span widths from column widths.
The following examples show the results from different \verb!hspan! values.

\begin{demohigh}
\SetTblrInner{hlines, vlines, hspan=default}
\begin{tblr}{cell{2}{1}={c=2}{l},cell{3}{1}={c=3}{l},cell{4}{2}={c=2}{l}}
 111 111 & 222 222 & 333 333 \\
 12 Multi Columns Multi Columns 12 & & 333 \\
 13 Multi Columns Multi Columns Multi Columns 13 & & \\
 111 & 23 Multi Columns Multi Columns 23 & \\
\end{tblr}
\end{demohigh}

\begin{demohigh}
\SetTblrInner{hlines, vlines, hspan=even}
\begin{tblr}{cell{2}{1}={c=2}{l},cell{3}{1}={c=3}{l},cell{4}{2}={c=2}{l}}
 111 111 & 222 222 & 333 333 \\
 12 Multi Columns Multi Columns 12 & & 333 \\
 13 Multi Columns Multi Columns Multi Columns 13 & & \\
 111 & 23 Multi Columns Multi Columns 23 & \\
\end{tblr}
\end{demohigh}

\begin{demohigh}
\SetTblrInner{hlines, vlines, hspan=minimal}
\begin{tblr}{cell{2}{1}={c=2}{l},cell{3}{1}={c=3}{l},cell{4}{2}={c=2}{l}}
 111 111 & 222 222 & 333 333 \\
 12 Multi Columns Multi Columns 12 & & 333 \\
 13 Multi Columns Multi Columns Multi Columns 13 & & \\
 111 & 23 Multi Columns Multi Columns 23 & \\
\end{tblr}
\end{demohigh}

The following examples show the results from different \verb!vspan! values.
\nopagebreak
\begin{demohigh}
\SetTblrInner{hlines, vlines, vspan=default}
\begin{tblr}{column{2}={3.25cm}, cell{2}{2}={r=3}{l}}
  Column1 & Column2 \\
  Row1 & Long text that needs multiple lines.
         Long text that needs multiple lines.
         Long text that needs multiple lines. \\
  Row2 & \\
  Row3 & \\
  Row4 & Short text \\
\end{tblr}
\end{demohigh}

\begin{demohigh}
\SetTblrInner{hlines, vlines, vspan=even}
\begin{tblr}{column{2}={3.25cm}, cell{2}{2}={r=3}{l}}
  Column1 & Column2 \\
  Row1 & Long text that needs multiple lines.
         Long text that needs multiple lines.
         Long text that needs multiple lines. \\
  Row2 & \\
  Row3 & \\
  Row4 & Short text \\
\end{tblr}
\end{demohigh}

\section{Default Specifications}

\verb!Tabularray! package provides \verb!\SetTblrInner! and \verb!\SetTblrOuter! commands
for you to change the default inner and outer specifications of tables.
Inner specifications are all options written in the mandatory argument of the \verb!tblr! environment,
while outer specifications are all options written in the optional argument of the \verb!tblr! environment.

In the below example, the first line draws all hlines and vlines for all tables created afterwards,
while the second line makes all tables created afterwards breakable.

\begin{codehigh}
\SetTblrInner{hlines,vlines}
\SetTblrOuter{long}
\end{codehigh}

You can define new \verb!tabularray! environments using \verb!\NewTblrEnviron! command:

\begin{codehigh}
\NewTblrEnviron{mylinetblr}{
  \SetTblrInner{hlines,vlines}
}
\NewTblrEnviron{mylongtblr}{
  \SetTblrOuter{long}
}
\end{codehigh}

\section{Counters in Tables}

Counters \verb!rownum!, \verb!colnum!, \verb!rowcount!, \verb!colcount! can be used in cell text:
\nopagebreak
\begin{demohigh}
\begin{tblr}{hlines}
 Cell[\arabic{rownum}][\arabic{colnum}] & Cell[\arabic{rownum}][\arabic{colnum}] &
 Cell[\arabic{rownum}][\arabic{colnum}] & Cell[\arabic{rownum}][\arabic{colnum}] \\
 Cell[\arabic{rownum}][\arabic{colnum}] & Cell[\arabic{rownum}][\arabic{colnum}] &
 Cell[\arabic{rownum}][\arabic{colnum}] & Cell[\arabic{rownum}][\arabic{colnum}] \\
 Row=\arabic{rowcount}, Col=\arabic{colcount} &
 Row=\arabic{rowcount}, Col=\arabic{colcount} &
 Row=\arabic{rowcount}, Col=\arabic{colcount} &
 Row=\arabic{rowcount}, Col=\arabic{colcount} \\
 Cell[\arabic{rownum}][\arabic{colnum}] & Cell[\arabic{rownum}][\arabic{colnum}] &
 Cell[\arabic{rownum}][\arabic{colnum}] & Cell[\arabic{rownum}][\arabic{colnum}] \\
 Cell[\arabic{rownum}][\arabic{colnum}] & Cell[\arabic{rownum}][\arabic{colnum}] &
 Cell[\arabic{rownum}][\arabic{colnum}] & Cell[\arabic{rownum}][\arabic{colnum}] \\
\end{tblr}
\end{demohigh}

\section{Tracing Tabularray}

\begin{tcolorbox}
The interfaces in this section should be seen as \experimental
and are likely to change in future releases, if necessary.
Don’t use them in important documents.
\end{tcolorbox}

To trace internal data behind \verb!tblr! environment, you can use \verb!\SetTblrTracing! command.
For example, \verb!\SetTblrTracing{all}! will turn on all tracings,
and \verb!\SetTblrTracing{none}! will turn off all tracings.
\verb!\SetTblrTracing{+row,+column}! will only tracing row and column data.
All tracing messages will be written to the log files.

\chapter{Use Long Tables}

\begin{tcolorbox}
The interfaces in this chapter should be seen as \experimental
and are likely to change in future releases, if necessary.
Don’t use them in important documents.
\end{tcolorbox}

\section{A Simple Example}

In fact, to make a decent long table with header and footer, it is better to separate header/footer as
\underline{table head/foot} (which includes caption, footnotes, continuation text)
and \underline{row head/foot} (which includes some rows of the table that should appear in every page).
By this approach, alternating row colors should work as expected.

\NewTblrTheme{fancy}{
  \SetTblrStyle{firsthead}{font=\bfseries}
  \SetTblrStyle{firstfoot}{fg=blue2}
  \SetTblrStyle{middlefoot}{\itshape}
  \SetTblrStyle{caption-tag}{red2}
}
\begin{longtblr}[
  theme = fancy,
  caption = {A Long Long Long Long Long Long Long Table},
  entry = {Short Caption},
  label = {tblr:test},
  note{a} = {It is the first footnote.},
  note{$\dag$} = {It is the second long long long long long long footnote.},
  remark{Note} = {Some general note. Some general note. Some general note.},
  remark{Source} = {Made up by myself. Made up by myself. Made up by myself.},
]{
  colspec = {XXX}, width = 0.85\linewidth,
  row{odd} = {gray9}, row{even} = {brown9}, rowhead = 2, rowfoot = 1,
}
\hline
 \SetRow{purple7} Head & Head & Head \\
\hline
 \SetRow{purple7} Head & Head & Head \\
\hline
 Alpha   & Beta  & Gamma   \\
\hline
 Epsilon & Zeta\TblrNote{a} & Eta \\
\hline
 Iota    & Kappa\TblrNote{$\dag$} & Lambda \\
\hline
 Nu      & Xi    & Omicron \\
\hline
 Rho     & Sigma & Tau     \\
\hline
 Phi     & Chi   & Psi     \\
\hline
 Alpha   & Beta  & Gamma   \\
\hline
 Epsilon & Zeta  & Eta     \\
\hline
 Iota    & Kappa & Lambda  \\
\hline
 Nu      & Xi    & Omicron \\
\hline
 Rho     & Sigma & Tau     \\
\hline
 Phi     & Chi   & Psi     \\
\hline
 Alpha   & Beta  & Gamma   \\
\hline
 Epsilon & Zeta  & Eta     \\
\hline
 Iota    & Kappa & Lambda  \\
\hline
 Nu      & Xi    & Omicron \\
\hline
 Rho     & Sigma & Tau     \\
\hline
 Phi     & Chi   & Psi     \\
\hline
 Alpha   & Beta  & Gamma   \\
\hline
 Epsilon & Zeta  & Eta     \\
\hline
 Iota    & Kappa & Lambda  \\
\hline
 Nu      & Xi    & Omicron \\
\hline
 Rho     & Sigma & Tau     \\
\hline
 Phi     & Chi   & Psi     \\
\hline
 Alpha   & Beta  & Gamma   \\
\hline
 Epsilon & Zeta  & Eta     \\
\hline
 Iota    & Kappa & Lambda  \\
\hline
 Nu      & Xi    & Omicron \\
\hline
 Rho     & Sigma & Tau     \\
\hline
 Phi     & Chi   & Psi     \\
\hline
 Alpha   & Beta  & Gamma   \\
\hline
 Epsilon & Zeta  & Eta     \\
\hline
 Iota    & Kappa & Lambda  \\
\hline
 Nu      & Xi    & Omicron \\
\hline
 Rho     & Sigma & Tau     \\
\hline
 Phi     & Chi   & Psi     \\
\hline
 Alpha   & Beta  & Gamma   \\
\hline
 Epsilon & Zeta  & Eta     \\
\hline
 Iota    & Kappa & Lambda  \\
\hline
 Nu      & Xi    & Omicron \\
\hline
 Rho     & Sigma & Tau     \\
\hline
 Phi     & Chi   & Psi     \\
\hline
 Alpha   & Beta  & Gamma   \\
\hline
 Epsilon & Zeta  & Eta     \\
\hline
 Iota    & Kappa & Lambda  \\
\hline
 Nu      & Xi    & Omicron \\
\hline
 Rho     & Sigma & Tau     \\
\hline
 Phi     & Chi   & Psi     \\
\hline
 Alpha   & Beta  & Gamma   \\
\hline
 Epsilon & Zeta  & Eta     \\
\hline
 Iota    & Kappa & Lambda  \\
\hline
 Nu      & Xi    & Omicron \\
\hline
 Rho     & Sigma & Tau     \\
\hline
 Phi     & Chi   & Psi     \\
\hline
 Alpha   & Beta  & Gamma   \\
\hline
 Epsilon & Zeta  & Eta     \\
\hline
 Iota    & Kappa & Lambda  \\
\hline
 Nu      & Xi    & Omicron \\
\hline
 Rho     & Sigma & Tau     \\
\hline
 Phi     & Chi   & Psi     \\
\hline
 Alpha   & Beta  & Gamma   \\
\hline
 Epsilon & Zeta  & Eta     \\
\hline
 Iota    & Kappa & Lambda  \\
\hline
 Nu      & Xi    & Omicron \\
\hline
 Rho     & Sigma & Tau     \\
\hline
 Phi     & Chi   & Psi     \\
\hline
Alpha   & Beta  & Gamma   \\
\hline
 Epsilon & Zeta  & Eta     \\
\hline
 Iota    & Kappa & Lambda  \\
\hline
 Nu      & Xi    & Omicron \\
\hline
 Rho     & Sigma & Tau     \\
\hline
 Phi     & Chi   & Psi     \\
\hline
 Alpha   & Beta  & Gamma\footnote{Hello}   \\
\hline
 Epsilon & Zeta  & Eta     \\
\hline
 Iota    & Kappa & Lambda  \\
\hline
 Nu      & Xi    & Omicron \\
\hline
 Rho     & Sigma & Tau     \\
\hline
 Phi     & Chi   & Psi     \\
\hline
 Alpha   & Beta  & Gamma   \\
\hline
 Epsilon & Zeta  & Eta     \\
\hline
 Iota    & Kappa & Lambda  \\
\hline
 Nu      & Xi    & Omicron \\
\hline
 Rho     & Sigma & Tau     \\
\hline
 Phi     & Chi   & Psi     \\
%\hline
% Alpha   & Beta  & Gamma   \\
%\hline
% Epsilon & Zeta  & Eta     \\
%\hline
% Iota    & Kappa & Lambda  \\
%\hline
% Nu      & Xi    & Omicron \\
%\hline
% Rho     & Sigma & Tau     \\
%\hline
% Phi     & Chi   & Psi     \\
\hline
 \SetRow{blue7} Foot & Foot & Foot \\
\hline
\end{longtblr}

As you can see in the above example, the appearance of long tables of \verb!tabularray! package
is similar to that of \verb!threeparttable! and \verb!threeparttablex! packages.
We support table footnotes, but not page footnotes in \verb!tabularray! package.

Table head consists of the caption and the last table foot consists of footnotes and remarks by default.
We can customize both of them to make them include other text, or appear in different pages.

\newpage

The source code for the above long table is shown below. It is mainly self-explanatory.

\begin{codehigh}
\NewTblrTheme{fancy}{
  \SetTblrStyle{firsthead}{font=\bfseries}
  \SetTblrStyle{firstfoot}{fg=blue2}
  \SetTblrStyle{middlefoot}{\itshape}
  \SetTblrStyle{caption-tag}{red2}
}
\begin{longtblr}[
  theme = fancy,
  caption = {A Long Long Long Long Long Long Long Table},
  entry = {Short Caption},
  label = {tblr:test},
  note{a} = {It is the first footnote.},
  note{$\dag$} = {It is the second long long long long long long footnote.},
  remark{Note} = {Some general note. Some general note. Some general note.},
  remark{Source} = {Made up by myself. Made up by myself. Made up by myself.},
]{
  colspec = {XXX}, width = 0.85\linewidth,
  row{odd} = {gray9}, row{even} = {brown9}, rowhead = 2, rowfoot = 1,
}
\hline
 \SetRow{purple7} Head & Head & Head \\
\hline
 \SetRow{purple7} Head & Head & Head \\
\hline
 Alpha   & Beta  & Gamma   \\
\hline
 Epsilon & Zeta\TblrNote{a} & Eta \\
\hline
 Iota    & Kappa\TblrNote{$\dag$} & Lambda \\
\hline
 Nu      & Xi    & Omicron \\
\hline
 Rho     & Sigma & Tau     \\
\hline
 Phi     & Chi   & Psi     \\
\hline
......
\hline
 Alpha   & Beta  & Gamma   \\
\hline
 Epsilon & Zeta  & Eta     \\
\hline
 Iota    & Kappa & Lambda  \\
\hline
 Nu      & Xi    & Omicron \\
\hline
 Rho     & Sigma & Tau     \\
\hline
 Phi     & Chi   & Psi     \\
\hline
 \SetRow{blue7} Foot & Foot & Foot \\
\hline
\end{longtblr}
\end{codehigh}

Row head and row foot consist of some lines of the table and should appear in every page.
In the above code, We add \verb!rowhead=2! and \verb!rowfoot=1! to set the total lines of row head and row foot.

\section{Customize Templates}

\subsection{Overview of Templates}

The template system for table heads and table foots in \verb!tabularray! is largely inspired by \verb!beamer!,
\verb!caption! and \verb!longtable! packages. You can use \verb!\DefTblrTemplate! command to define and
modify templates, and \verb!\SetTblrTemplate! command to choose default templates. In defining templates,
you can include other templates with \verb!\UseTblrTemplate! and \verb!\ExpTblrTemplate! commands.

\begin{longtblr}[
  caption = {Templates for Table Heads and Table Foots}
]{
  hlines = {white},
  row{odd}={brown8},row{even}={gray8},row{1}={fg=white,bg=purple2,font=\bfseries\sffamily},
  column{2} = {co=1},
  rowhead = 1,
}
  Template Name    & Template Description \\
  \V{contfoot-text}& continuation text in the foot, normally ``Continued on next page'' \\
  \V{contfoot}     & continuation paragraph in the foot, normally including \V{contfoot-text} template \\
  \V{conthead-text}& continuation text in the head, normally ``(Continued)'' \\
  \V{conthead}     & continuation paragraph in the head, normally including \V{conthead-text} template \\
  \V{caption-tag}  & caption tag, normally like ``Table 4.2'' \\
  \V{caption-sep}  & caption separator, normally like ``:\quad'' \\
  \V{caption-text} & caption text, normally using user provided value \\
  \V{caption}      & including \V{caption-tag} + \V{caption-sep} + \V{caption-text} \\
  \V{note-tag}     & note tag, normally using user provided value \\
  \V{note-sep}     & note separator, normally like ``\enskip'' \\
  \V{note-text}    & note tag, normally using user provided value \\
  \V{note}         & including \V{note-tag} + \V{note-sep} + \V{note-text} \\
  \V{remark-tag}   & remark tag, normally using user provided value \\
  \V{remark-sep}   & remark separator, normally like ``:\enskip'' \\
  \V{remark-text}  & remark text, normally using user provided value\\
  \V{remark}       & including \V{remark-tag} + \V{remark-sep} + \V{remark-text} \\
  \V{firsthead}    & table head on the first page, normally including \V{caption} template \\
  \V{middlehead}   & table head on middle pages, normally including \V{caption} and \V{conthead} templates \\
  \V{lasthead}     & table head on the last page, normally including \V{caption} and \V{conthead} templates \\
  \V{head}         & setting all of \V{firsthead}, \V{middlehead} and \V{lasthead} \\
  \V{firstfoot}    & table foot on the first page, normally including \V{contfoot} template \\
  \V{middlefoot}   & table foot on middle pages, normally including \V{contfoot} template \\
  \V{lastfoot}     & table foot on the last page, normally including \V{note} and \V{remark} templates \\
  \V{foot}         & setting all of \V{firstfoot}, \V{middlefoot} and \V{lastfoot} \\
\end{longtblr}

A template which only includes short text is called a \underline{sub template}.
Normally there is one \verb!-! in the name of a sub template.
A template which includes one or more paragraphs is called a \underline{main template}.
Normally there isn't any \verb!-! in the name of a main template.

\subsection{Continuation Templates}

Let us have a look at the code for defining templates of continuation text first:

\begin{codehigh}
\DefTblrTemplate{contfoot-text}{normal}{Continued on next page}
\SetTblrTemplate{contfoot-text}{normal}
\DefTblrTemplate{conthead-text}{normal}{(Continued)}
\SetTblrTemplate{conthead-text}{normal}
\end{codehigh}

In the above code, command \verb!\DefTblrTemplate! defines the templates with name \verb!normal!,
and then command \verb!\SetTblrTemplate! sets the templates with name \verb!normal! as default.
The \verb!normal! template is always defined and set as default for any template element in \verb!tabularray!.
Therefore you had better use another name when defining new templates.

If you use \verb!default! as template name in \verb!\DefTblrTemplate!,
you define and set it as default at the same time.
Therefore the above code can be written in another way:

\begin{codehigh}
\DefTblrTemplate{contfoot-text}{default}{Continued on next page}
\DefTblrTemplate{conthead-text}{default}{(Continued)}
\end{codehigh}

You may modify the code to customize continuation text to fit your needs.

The \verb!contfoot! and \verb!conthead! templates normally
include their sub templates with \verb!\UseTblrTemplate! commands.
But you can also handle user settings such as horizontal alignment here.

\begin{codehigh}
\DefTblrTemplate{contfoot}{default}{\UseTblrTemplate{contfoot-text}{default}}
\DefTblrTemplate{conthead}{default}{\UseTblrTemplate{conthead-text}{default}}
\end{codehigh}

\subsection{Caption Templates}

Normally a caption consists of three parts, and their templates are defined with the follow code:

\begin{codehigh}
\DefTblrTemplate{caption-tag}{default}{Table\hspace{0.25em}\thetable}
\DefTblrTemplate{caption-sep}{default}{:\enskip}
\DefTblrTemplate{caption-text}{default}{\InsertTblrText{caption}}
\end{codehigh}

The command \verb!\InsertTblrText{caption}! inserts the value of \verb!caption! key,
which you could write in the optional argument of \verb!longtblr! environment.

The \verb!caption! template normally includes three sub templates with \verb!\UseTblrTemplate! commands:
The \verb!caption! template will be used in \verb!firsthead! template.

\begin{codehigh}
\DefTblrTemplate{caption}{default}{
  \UseTblrTemplate{caption-tag}{default}
  \UseTblrTemplate{caption-sep}{default}
  \UseTblrTemplate{caption-text}{default}
}
\end{codehigh}

Furthermore \verb!capcont! template includes \verb!conthead! template as well.
The \verb!capcont! template will be used in \verb!middlehead! and \verb!lasthead! templates.

\begin{codehigh}
\DefTblrTemplate{capcont}{default}{
  \UseTblrTemplate{caption-tag}{default}
  \UseTblrTemplate{caption-sep}{default}
  \UseTblrTemplate{caption-text}{default}
  \UseTblrTemplate{conthead}{default}
}
\end{codehigh}

\subsection{Note and Remark Templates}

The templates for table notes can be defined like this:

\begin{codehigh}
\DefTblrTemplate{note-tag}{default}{\textsuperscript{\InsertTblrNoteTag}}
\DefTblrTemplate{note-sep}{default}{\space}
\DefTblrTemplate{note-text}{default}{\InsertTblrNoteText}
\end{codehigh}
\begin{codehigh}
\DefTblrTemplate{note}{default}{
  \MapTblrNotes{
    \noindent
    \UseTblrTemplate{note-tag}{default}
    \UseTblrTemplate{note-sep}{default}
    \UseTblrTemplate{note-text}{default}
    \par
  }
}
\end{codehigh}

The \verb!\MapTblrNotes! command loops for all table notes,
which are written in the optional argument of \verb!longtblr! environment.
Inside the loop, you can use \verb!\InsertTblrNoteTag! and \verb!\InsertTblrNoteText!
commands to insert current note tag and note text, respectively.

The definition of remark templates are similar to note templates.
\nopagebreak
\begin{codehigh}
\DefTblrTemplate{remark-tag}{default}{\InsertTblrRemarkTag}
\DefTblrTemplate{remark-sep}{default}{:\space}
\DefTblrTemplate{remark-text}{default}{\InsertTblrRemarkText}
\end{codehigh}
\begin{codehigh}
\DefTblrTemplate{remark}{default}{
  \MapTblrRemarks{
    \noindent
    \UseTblrTemplate{remark-tag}{default}
    \UseTblrTemplate{remark-sep}{default}
    \UseTblrTemplate{remark-text}{default}
    \par
  }
}
\end{codehigh}

\subsection{Head and Foot Templates}

The templates for table heads and foots are defined as including other templates:

\begin{codehigh}
\DefTblrTemplate{firsthead}{default}{
  \UseTblrTemplate{caption}{default}
}
\DefTblrTemplate{middlehead,lasthead}{default}{
  \UseTblrTemplate{capcont}{default}
}
\DefTblrTemplate{firstfoot,middlefoot}{default}{
  \UseTblrTemplate{contfoot}{default}
}
\DefTblrTemplate{lastfoot}{default}{
  \UseTblrTemplate{note}{default}
  \UseTblrTemplate{remark}{default}
}
\end{codehigh}

Note that you can define the same template for multiple elements in \verb!\DefTblrTemplate! command.

\section{Change Styles}

You may change the styles of template elements.

\begin{codehigh}
\SetTblrStyle{firsthead}{font=\bfseries}
\SetTblrStyle{firstfoot}{fg=blue2}
\SetTblrStyle{middlefoot}{\itshape}
\SetTblrStyle{caption-tag}{red2}
\end{codehigh}

\newpage %TODO: fix pagebreak bug

All available keys for template elements are described in Table \ref{key:element}.

\begin{longtblr}[
  caption = {Keys for the Styles of Template Elements},
  label = {key:element},
  remark{Note} = {In most cases, you can omit the underlined key names and write only their values.
                  The keys \K{halign}, \K{indent} and \K{hang} are only for main templates.}
]{
  hlines = {white}, rowhead = 1,
  row{odd}={brown8},row{even}={gray8},row{1}={fg=white,bg=purple2,font=\bfseries\sffamily},
  column{2} = {co=1},
}
  Key Name               & Key Description \\
  \underline{\K{fg}}     & foreground color \\
  \underline{\K{font}}   & font commands \\
  \underline{\K{halign}} & horizontal alignment: \V{l} (left), \V{c} (center), or \V{r} (right) \\
  \K{indent}             & parindent value \\
  \K{hang}               & hangindent value \\
\end{longtblr}

When you write \verb!\UseTblrTemplate{element}{default}! in defining a template,
beside including the code of template \verb!element!, the foreground color and font commands
of the template \verb!element! will be set up automatically.
In contrast, \verb!\ExpTblrTemplate{element}{default}! will only include template code.

\section{Define Themes}

You may define your own themes for table heads and foots with \verb!\NewTblrTheme! command.
a theme consists of some template and style settings. For example:
\nopagebreak
\begin{codehigh}
\NewTblrTheme{fancy}{
  \DefTblrTemplate{conthead}{default}{[Continued]}
  \SetTblrStyle{firsthead}{font=\bfseries}
  \SetTblrStyle{firstfoot}{fg=blue2}
  \SetTblrStyle{middlefoot}{\itshape}
  \SetTblrStyle{caption-tag}{red2}
}
\end{codehigh}

After defining the theme \verb!fancy!, you can use it
by writing \verb!theme=fancy! in the optional argument of \verb!longtblr! environment.

\chapter{Use Some Libraries}

\begin{tcolorbox}
The interfaces in this chapter should be seen as \experimental
and are likely to change in future releases, if necessary.
Don’t use them in important documents.
\end{tcolorbox}

The \verb!tabularray! package emulates or fixes some commands in other packages.
To avoid potential conflict, you need to enable them with \verb!\UseTblrLibrary! command.

\section{Library \V{booktabs}}

When using \verb!\UseTblrLibrary{booktabs}!,
\verb!tabularray! package defines commands \verb!\toprule!, \verb!\midrule! and \verb!\bottomrule!
inside \verb!tblr! environment.

\begin{demohigh}
\begin{tblr}{llll}
\toprule
 Alpha   & Beta  & Gamma   & Delta \\
\midrule
 Epsilon & Zeta  & Eta     & Theta \\
 Iota    & Kappa & Lambda  & Mu    \\
 Nu      & Xi    & Omicron & Pi    \\
\bottomrule
\end{tblr}
\end{demohigh}

\section{Library \V{diagbox}}

When using \verb!\UseTblrLibrary{diagbox}! in the preamble of the document,
\verb!tabularray! package load \verb!diagbox! package,
and you can use \verb!\diagbox! and \verb!\diagboxthree! commands inside \verb!tblr! environment.

\begin{demohigh}
\begin{tblr}{hlines,vlines}
 \diagbox{Aa}{Pp} & Beta & Gamma \\
 Epsilon & Zeta  & Eta \\
 Iota    & Kappa & Lambda \\
\end{tblr}
\end{demohigh}

\begin{demohigh}
\begin{tblr}{hlines,vlines}
 \diagboxthree{Aa}{Pp}{Hh} & Beta & Gamma \\
 Epsilon & Zeta  & Eta \\
 Iota    & Kappa & Lambda \\
\end{tblr}
\end{demohigh}

You can also use \verb!\diagbox! and \verb!\diagboxthree! commands in math mode.
\nopagebreak
\begin{demohigh}
$\begin{tblr}{|c|cc|}
\hline
 \diagbox{X_1}{X_2} & 0 & 1 \\
\hline
  0 & 0.1 & 0.2 \\
  1 & 0.3 & 0.4 \\
\hline
\end{tblr}$
\end{demohigh}

\chapter{The Source Code}

%\CodeHigh{lite}
\dochighinput[language=latex/latex3]{tabularray.sty}

\end{document}
